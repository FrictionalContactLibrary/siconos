


\begin{tabular}{lll}
  \centering
  Author &  O. Bonnefon &2010\\
  Revision& section \ref{Sec:NE_motion} to \ref{Sec:NE_TD} V. Acary&  05/09/2011\\
  Revision& section \ref{Sec:NE_motion}  V. Acary&  01/06/2016\\
  Revision& complete edition V. Acary&  06/01/2017\\

\end{tabular}


\section{The equations of motion}


\def\cg{\sf \small g}
In the maximal coordinates framework, the most natural choice for the kinematic  variables and for the formulation of the equations of motion is the Newton/Euler formalism, where the equation of motion describes the translational and rotational dynamics of each body using a specific choice of parameters. For the translational motion, the position of the center of mass $x_{\cg}\in \RR^3$ and its velocity  $v_{\cg} = \dot x_{\cg} \in \RR^3$ is usually chosen. For the orientation of the body is usually defined by the rotation matrix $R$ of the body-fixed frame with respect to a given inertial frame.

For the rotational motion, a common choice is to choose the rotational velocity  $\Omega \in \RR^3$ of the body expressed in the body--fixed frame. This choice comes from the formulation of a rigid body motion of a point $X$ in the inertial frame as
\begin{equation}
  \label{eq:28}
  x(t) = \Phi(t,X) = x_{\cg}(t) + R(t) X.
\end{equation}
The velocity of this point can be written as
\begin{equation}
  \label{eq:29}
  \dot x(t) = v_{\cg} + \dot R(t) X
\end{equation}
Since $R^\top R=I$, we get $R^\top \dot R + \dot R^\top R =0$. We can conclude that it exists a matrix $\tilde \Omega := R^\top \dot R $ such that $\tilde \Omega + \tilde \Omega^\top=0$, i.e. a skew symmetric matrix. The notation $\tilde \Omega$ comes from the fact that there is a bijection between the skew symmetric matrix in $\RR^{3\times3}$ and $\RR^3$ such that
\begin{equation}
  \label{eq:30}
  \tilde \Omega x  = \Omega \times x, \quad \forall x\in RR^3.
\end{equation}
The rotational velocity is then related to the $R$ by :
\begin{equation}
  \label{eq:angularvelocity}
  \widetilde \Omega = R^\top \dot R, \text { or equivalently, } \dot R  = R \widetilde \Omega
\end{equation}



Using these coordinates, the equations of motion are given by 
\begin{equation}
  \label{eq:motion-NewtonEuler}
  \left\{\begin{array}{rcl}
      m \;\dot v_{\cg}  & = &f(t,x_{\cg}, v_{\cg},  R,  \Omega) \\
      I \dot \Omega + \Omega \times I \Omega &= & M(t,x_{\cg}, v_{\cg}, R, \Omega) \\
      \dot x_{\cg}&=& v_{\cg}\\
      \dot R  &=& R \widetilde \Omega
    \end{array}
  \right.
\end{equation}
where $m> 0$ is the mass, $I\in \RR^{3\times 3}$ is the matrix of moments of inertia around the center of mass and the axis of the body--fixed frame.

The vectors $f(\cdot)\in \RR^3$ and $M(\cdot)\in \RR^3$ are the total forces and torques applied to the body. It is important to outline that the total applied forces $f(\cdot)$ has to be expressed in a consistent frame w.r.t. to $v_{\cg}$. In our case, it hae to be expressed in the inertial frame. The same applies for the moment $M$ that has to be expressed in the body-fixed frame. If we consider a moment $m(\cdot)$ expressed in the inertial frame, then is has to be convected to  the body--fixed frame thanks to
\begin{equation}
  \label{eq:convected_moment}
  M (\cdot) =R^\top  m (\cdot)
\end{equation}


\begin{remark}
If we perform the time derivation of $RR^\top =I$ rather than $R^\top R=I$, we get $R \dot R^\top + \dot R R^\top =0$.  We can conclude that it exists a matrix $\tilde \omega := \dot R R^\top $ such that $\tilde \omega + \tilde \omega^\top=0$, i.e. a skew symmetric matrix. Clearly, we have
 \begin{equation}
   \label{eq:31}
   \tilde \omega = R \tilde \Omega R^\top
 \end{equation}
 and it can be proved that is equivalent to $ \omega =R \Omega$. The vector $\omega$ is the rotational velocity expressed in the inertial frame.
\end{remark}
 
\begin{ndrva}
  todo : add the formulation in the inertial frame of the Euler equation with $\omega =R \Omega$.
\end{ndrva}

\section{Playing with rotations and quaternions}
\subsection{ Lie group $SO(3)$ of finite rotations and Lie algebra $\mathfrak{so}(3)$ of infinitesimal rotations}

Recall that the Lie group $SO(3)$ of linear proper orthogonal transformation by its matrix representation
\begin{equation}SO(3)
  \label{eq:43}
  SO(3) = \{R \in \RR^{3\times3}\mid R^TR=I , det(R) = +1  \}
\end{equation}
with the Lie group composition $( \circ )$ law given by $R_1\circ R_2 = R_1R_2$ for $R_1,R_2\in SO(3)$. The identity element is $e = I_{3\times 3}$. At any point of $R\in SO(3)$, the tangent space is noted $T_RSO(3)$ is the set of tangent vectors at a point $R$. Given a smooth curve $S(\cdot) : t\in \RR \mapsto S(t)\in SO(3)$ in $SO(3)$ such that $S(O)= R$, a tangent vector at $R$ is defined by
\begin{equation}
  \label{eq:59}
  a = \left. \frac{d}{dt} (S(t)) \right|_{t=0}.
\end{equation}

\paragraph{Left representation of  the tangent space at $R$, $T_RSO(3)$ }
Since $S(t)\in SO(3)$, we have $S(t)S(t)^T=I$ and then $\frac{d}{dt} (S(t)) = \dot S(t)S^T(t) +  S(t) \dot S^T(t) =0$. At $t=0$, we get
\begin{equation}
  \label{eq:60}
 a R^T +  R a^T =0.
\end{equation}
We conclude that it exists a skew--symmetric matrix $\tilde \Omega = R^T a$ such that $\tilde \Omega^T + \tilde \Omega =0$. Therefore a possible representation of  $T_RSO(3)$ is
\begin{equation}
  \label{eq:58}
  T_RSO(3) = \{ a = R \tilde \Omega \in \RR^{3\times 3} \mid \tilde \Omega^T + \tilde \Omega =0 \}.
\end{equation}

For $R=I$, the tangent space is directly given by the set of  skew--symmetric matrices:
\begin{equation}
  \label{eq:58}
  T_ISO(3) = \{ a = \tilde \Omega\in \RR^{3\times 3} \mid \tilde \Omega^T + \tilde \Omega =0 \}.
\end{equation}
Firstly, it is possible to show that $T_ISO(3)$ with the Lie Bracket $[\cdot,\cdot]$ defined by the matrix commutator
\begin{equation}
  \label{eq:61}
  [A,B] = AB-BA
\end{equation}
is a Lie algebra. For skew symmetric matrices, the commutator can be expressed with the cross product in $\RR^3$
\begin{equation}
  \label{eq:62}
  [\tilde \Omega, \tilde \Gamma] = \tilde \Omega \tilde \Gamma - \tilde \Gamma \tilde \Omega = \widetilde{ \Omega \Gamma}
\end{equation}
The specific notation $\mathfrak{so}(3)$  is generally introduced for the Lie algebra at the identity point
\begin{equation}
  \label{eq:44}
  \mathfrak{so}(3) =\{\Omega\in \RR^{3\times 3} \mid \Omega + \tilde \Omega^T =0\}.
\end{equation}
and we use   $T_ISO(3) =  \mathfrak{so}(3)$ whenever there is no ambiguity  The notation $\tilde \Omega$ is implied by the fact that the Lie algebra is isomorphic to $\RR^3$ thanks to the operator $\widetilde{(\cdot)} :\RR^3 \rightarrow \mathfrak{so}(3)$ and defined by
\begin{equation}
  \label{eq:45}
 \widetilde{(\cdot)}: \Omega \mapsto \tilde \Omega =
  \begin{bmatrix}
    0 & -\Omega_3 & \Omega_2 \\
    \Omega_3 & 0 & -\Omega_1 \\
    -\Omega_2  & \Omega_1 & 0
  \end{bmatrix}
\end{equation}
Note that $\tilde \Omega x = \Omega \times x$. With this notation the commutator can be expressed as
\begin{equation}
  \label{eq:63}
  [\tilde \Omega, \tilde \Gamma] = \Omega \times \Gamma
\end{equation}

Let us come back to the representation of  $T_RSO(3)$ given in~\eqref{eq:58}. It is clear it can expressed with a representation that relies on $\mathfrak{so}(3)$
\begin{equation}
  \label{eq:64}
   T_RSO(3) = \{ a = R \tilde \Omega \in \RR^{3\times 3} \mid \tilde \Omega \in \mathfrak{so}(3) \}.
\end{equation}
With \eqref{eq:64}, we see that there a linear map that relates $T_RSO(3)$ to  $\mathfrak{so}(3)$.

Let us recall the definition of a action on a Lie group. An general (left)  action of Lie Group $\mathcal G$ on a manifold $\mathcal M$ is a smooth map $\Lambda: \mathcal G \times \mathcal \mathcal M \rightarrow \mathcal M$ satisfying
\begin{equation}
  \label{eq:66}
  \begin{array}[lcl]{rcl}
    \Lambda(I,y) &=& y, \quad \forall y \in \mathcal M \\
    \Lambda(R,\Lambda(S,y)) &=& \Lambda(RS,y) , \quad \forall R,S \in \mathcal G  \forall y \in \mathcal M \\
  \end{array}
\end{equation}
A special group action that is  the left translation map for a point $R \in SO(3)$ 
\begin{equation}
  \label{eq:65}
  \begin{array}[lcl]{rcl}
    L_R& :  & SO(3)  \rightarrow  SO(3)\\
            & & R \times S  \mapsto L_R(S) = R \circ S = RS\\
  \end{array}
\end{equation}
which is diffeomorphism on $SO(3)$. In that case, we identify the manifold and the group.

A given smooth curve  $S(\cdot) : t\in \RR \mapsto S(t)\in \mathcal G$ in $\mathcal G$ such that $S(0)= I$ produces a flow $\Lambda(S(t),cdot)$ on $\mathcal M$ and by differentiation we find a vector field
\begin{equation}
  \label{eq:67}
    F(y) = \left. \frac{d}{dt} (\Lambda(S(t),y) \right|_{t=0}.
\end{equation}
For our application where  $\mathcal G = \mathcal M = SO(3)$, we get
\begin{equation}
  \label{eq:68}
  F(y) = \left. \frac{d}{dt} S(t)\circ y \right|_{t=0} = \dot S(0) \circ y \in X(\mathcal M)
\end{equation}
Since $S(\cdot)$ is a smooth curve in $\mathcal G$, $\dot S(0)$ is a tangent vector at the point $S(0)=I$, that is an element $a = \tilde \Omega  \in \mathfrak{so}(3) $ defined by the relation~\eqref{eq:59}. Therefore, the vector field in \eqref{eq:68} is a tangent vector field and we get
\begin{equation}
  \label{eq:70}
  \dot y(t) = F(y(t)) = \tilde \Omega y(t) = \Omega \times y(t).
\end{equation}

\begin{ndrva}
  what happens at $S(0)=R$, with $ a =R \tilde \Omega =\dot S(0)$ and then $\dot y(t) = F(y(t)) = R \tilde \Omega y(t) =  R\Omega \times y(t)= \dot S(0) y(t) $. What else ? 
\end{ndrva}


\paragraph{Exponential map}
Let us now define $\lambda_* :  \mathfrak{so}(3) \rightarrow \mathcal X(\mathcal M)$ be defined as
\begin{equation}
  \label{eq:69}
  \lambda_*(a)(y) = \left. \frac{d}{ds} (\Lambda(S(s),y) \right|_{s=0}
\end{equation}
for a smooth curve  $S(s) \in \mathcal G$ such that $S(0) =I$ and $S'(0)=a$.


\paragraph{Adjoint representation}


Its derivative with respect to $S$, $DL_R(S)$ defines a diffeomorphism between $T_SSO(3)$ and $T_{RS}SO(3)$. In particular, for $S=I_{3\times3}$, we get
\begin{equation}
  \label{eq:49}
  \begin{array}{rcl}
    DL_R(I_{3\times3}) : \mathfrak{so}(3) & \rightarrow & T_R SO(3) \\
    \tilde \Omega &\mapsto &DL_R(I_{3\times3})\cdot \tilde \Omega = R \tilde \Omega
  \end{array}
\end{equation}
We end up with a possible representation of $T_{R} SO(3)$ as
\begin{equation}
  \label{eq:48}
  T_{R} SO(3) =\{\tilde \Omega_R \mid \tilde \Omega_R = DL_R(I_{3\times3})\cdot \tilde \Omega = R \tilde \Omega, \tilde \Omega \in\mathfrak{so}(3)  \}.
\end{equation}
In other words, a tangent vector $\tilde \Omega \in \mathfrak{so}(3)$ defines a left invariant vector field on $SO(3)$ at the point $R$ given by $R \tilde \Omega$.



\begin{ndrva}
  explain briefly the notion of left-invariant vector field 
\end{ndrva}
%\href{https://en.wikipedia.org/wiki/Lie_group}{https://en.wikipedia.org/wiki/Lie_group}
%Finally, the straight line $\alpha \tilde \Omega$ for $\Omega$ 
For a given $\tilde \Omega \in \mathfrak{so}(3)=T_{I_{3\times3}} SO(3) $, let us consider the differential equation
\begin{equation}
  \label{eq:50}
  \begin{cases}
    \dot R(t) = R \tilde \Omega\\
    R(0) = I
  \end{cases}
\end{equation}
Since $R \tilde \Omega$ defines a left invariant vector field on $SO(3)$, the solution of this differential equation will stay in $SO(3)$. Let us recall the definition of the matrix exponential,
\begin{equation}
  \label{eq:53}
  \exp(A) = \sum_{k=0}^{\infty} \frac {1}{k!} A^k
\end{equation}
A trivial solution of \eqref{eq:50} is $R(t) = \exp(t\tilde\Omega) $ since
\begin{equation}
  \label{eq:51}
  \frac {d}{dt}(\exp(At)) = \exp(At) A.
\end{equation}
More generally, with the initial condition $R(t_0)= R_0$, we get the solution
\begin{equation}
R(t) = R_0 \exp((t-t_0)\tilde\Omega)\label{eq:54}
\end{equation}



Another interpretation is as follows. From a (incremental) rotation vector, $\Omega$ and its associated matrix $\tilde \Omega$, we obtain a rotation matrix in $SO(3)$ by the exponentation of the tangent vector field:
\begin{equation}
  \label{eq:52}
  R = \exp(\tilde\Omega).
\end{equation}
Since we note that $\tilde \Omega^ 3 = - \theta^2 \tilde \Omega$ with $\theta = \|\Omega\|$, it is possible to get a closed form of the matrix exponential of $\tilde \Omega$
\begin{equation}
  \label{eq:55}
  \begin{array}[lcl]{lcl}
     \exp(\tilde \Omega) &=& \sum_{k=0}^{\infty} \frac {1}{k!} (\tilde \Omega)^k \\
                         &=&  I_{3\times 3} + \sum_{k=0}^{\infty} \frac {(-1)^k}{(2k+1)!} ( \theta)^{2k+1} \tilde \Omega + (\sum_{k=0}^{\infty} \frac {(-1)^k}{(2k)!} ( \theta)^{2k} -1)\tilde \Omega^2\\
                         &=&  I_{3\times 3} +\sin{\theta}\tilde \Omega +  (\cos(\theta)-1)\tilde \Omega^2 \\                   
  \end{array} 
\end{equation}
that is
\begin{equation}
  \label{eq:56}
  R = I_{3\times 3} +\sin{\theta}\tilde \Omega +  (\cos(\theta)-1)\tilde \Omega^2.
\end{equation}
The formula \eqref{eq:56} is the Euler--Rodrigues formula that allows to compute the rotation matrix on closed form.

The differential of the exponential mapping, denoted by $D_A\exp$ is defined as the 'right trivialized' tangent of the exponential map (see Definition 2.18 in~\cite{Iserles.ea_AN2000})
\begin{equation}
  \label{eq:57}
  \frac{d}{dt} (\exp(A(t))) = D_A\exp(A'(t)) \exp(A(t))
\end{equation}

\section{Parametrization using quaternions}


In the numerical practice, the choice of the rotation matrix is not convenient since it introduces redundant parameters. Since $R$ must belong to $SO^+(3)$, we have also to satisfy $\det(R)=1$ and $R^{-1}=R^\top$. In general, we use a reduced vector of parameters $p\in\RR^{n_p}$ such $R = \Phi(p)$ and $\dot p = \psi(p)\Omega $. We denote  by $q$ the vector of coordinates of the position and the orientation of the body, and by $v$ {the body twist}:
\begin{equation}
  q \coloneqq \begin{bmatrix}
    x_{\cg}\\
    p
  \end{bmatrix},\quad 
  v \coloneqq \begin{bmatrix}
     v_{\cg}\\
     \Omega
   \end{bmatrix}.
 \end{equation}
 The relation between $v$ and the time derivative of $q$ is
\begin{equation}
  \label{eq:TT}
  \dot q = 
  \begin{bmatrix}
     \dot x_{\cg}\\
     \psi(p) \dot p
   \end{bmatrix}
   = 
   \begin{bmatrix}
     I & 0 \\
     0 & \psi(p)
   \end{bmatrix}
   v
   \coloneqq
   T(q) v
\end{equation}
with $T(q) \in \RR^{7\times 6}$.
{Note that the twist $v$ is not directly the time derivative of the coordinate vector as a major difference with Lagrangian systems. }

\paragraph{Quaternions and rotation} 

In Siconos we choose to parametrize the rotation with a unit quaternion $p \in \RR^4$ such that $R = \Phi(p)$ and $\dot p = 1/2 \Psi(p)\Omega $. This parameterization has no singularity and has only one redundant variable that is determined by imposing $\|p\|=1$.
 Let us recall some basic properties of a quaternion $p$ that is a collection of $4$ real parameters $(p_0,p_1,p_2,p_3)$. The first is considered as a scalar and three last as a vector in $\RR^3$ denoted by $\vv{p}$. The quaternion can written as $p=(p_0,p_1,p_2,p_3)$ or $p=(p_0,\vv{p})$. The norm of a quaternion is given by $|p|^2=p^\top p = p_o^2+p_1^2+p_2^2+p_3^2$.
 With this notation, the quaternion product is defined by
\begin{equation}
  \label{eq:25}
  p \circ q =
  \begin{bmatrix}
    p_oq_o - \vv{p}\vv{q} \\
    p_0\vv{q}+q_o\vv{p} + \vv{p}\times\vv{q}
  \end{bmatrix}.
\end{equation}
The inverse element for the quaternion product is $e= (1,\vv{0})$. Let us note that 
\begin{equation}
  \label{eq:36}
  (0,\vv{p})\circ (0,\vv(q)) = - (0,\vv{q})\circ (0,\vv(p)).
\end{equation}
The quaternion multiplication can also be represented as a matrix operation in $\RR^{4\times4}$. Indeed, we have
\begin{equation}
  \label{eq:26}
  p \circ q  =
  \begin{bmatrix}
    q_0 p_0 -q_1p_1-q_2p_2-q_3p_3\\
    q_0 p_1 +q_1p_0-q_2p_3+q_3p_2\\
    q_0 p_2 +q_1p_3+q_2p_0-q_3p_1\\
    q_0 p_3 -q_1p_2+q_2p_1+q_3p_0\\
  \end{bmatrix}
\end{equation}
that can be represented as
\begin{equation}
  \label{eq:26}
  p \circ q  =
  \begin{bmatrix}
    p_0 & -p_1 & -p_2 & -p_3 \\
    p_1 & p_0 & -p_3 & p_2 \\
    p_2 & p_3 & p_0 & -p_1 \\
    p_3 & -p_2 & p_1 & p_0 \\
  \end{bmatrix}
  \begin{bmatrix}
    q_0\\
    q_1\\
    q_2\\
    q_3
  \end{bmatrix} := [p_\circ]q
\end{equation}
or
\begin{equation}
  \label{eq:26}
  p \circ q  = 
  \begin{bmatrix}
    q_0 & -q_1 & -q_2 & -q_3 \\
    q_1 & q_0 & q_3 & -q_2 \\
    q_2 & -q_3 & q_0 & q_1 \\
    q_3 & q_2 & -q_1 & q_0 \\
  \end{bmatrix}
  \begin{bmatrix}
    p_0\\
    p_1\\
    p_2\\
    p_3
  \end{bmatrix} := [{}_\circ q] p
\end{equation}

The adjoint quaternion of $p$ is denoted by
\begin{equation}
  \bar p =
  \begin{pmatrix}
    p_0, -p_1, -p_2, -p_3
  \end{pmatrix} = (p_0, - \vv{p})
\end{equation}
In particular, we have $p \circ \bar p = \bar p \circ p = |p|^2 e$. This allows to define the reciprocal of a non zero quaternion by
\begin{equation}
  \label{eq:32}
  p ^{-1} = \frac 1 {|p|^2} \bar p
\end{equation}

\paragraph{Unit quaternion and rotation}
A quaternion $p$is said to be unit if $|p| =1$. For two vectors $x\in \RR^3$ and $x'\in \RR^3$, we define the quaternion $p_x = (0,x)$ and  $p_{x'} = (0,x')$.
For a given unit quaternion $p$, the transformation
\begin{equation}
  \label{eq:20}
  p_{x'} = p \circ p_x \circ \bar p 
\end{equation}
defines a rotation $R$ such that $x'  = R x$ given by
\begin{equation}
  \label{eq:21}
  x' = (p_0^2- p^\top \vv{p}) x +2 p_0(\vv{p}\times x) +  2 (\vv{p}^\top x) p = R x
\end{equation}
The rotation matrix may be computed as 
\begin{equation}
  \label{eq:19}
  R = \Phi(p) =
  \begin{bmatrix}
    1-2 p_2^2- 2 p_3^2 & 2(p_1p_2-p_3p_0) & 2(p_1p_3+p_2p_0)\\
    2(p_1p_2+p_3p_0) & 1-2 p_1^2- 2 p_3^2 & 2(p_2p_3-p_1p_0)\\
    2(p_1p_3-p_2p_0) & 2(p_2p_3+p_1p_0)  & 1-2 p_1^2- 2 p_2^2\\
  \end{bmatrix}
\end{equation} 


\paragraph{Computation of the time derivative of a unit  quaternion associated with a rotation.}
The derivation with respect to time can obtained as follows. The rotation transformation for a unit quaternion is given by
\begin{equation}
  \label{eq:33}
  p_{x'}(t) = p(t) \circ p_x \circ \bar p(t) =  p(t) \circ p_x \circ p^{-1}(t)
\end{equation}
and can be derived as
\begin{equation}
  \label{eq:33}
  \begin{array}{lcl}
    \dot p_{x'}(t) &=& \dot p(t) \circ p_x \circ p^{-1}(t) + p(t) \circ p_x \circ \dot p^{-1}(t) \\
                  &=& \dot p(t) \circ p^{-1}(t)  \circ   p_{x'}(t)  +      p_{x'}(t) \circ p(t)  \circ \dot p^{-1}(t)    
  \end{array}
\end{equation}
From $p(t) \circ p^{-1}(t) =e$, we get
\begin{equation}
  \label{eq:34}
  \dot p(t) \circ p^{-1}(t) + p \circ \dot p^{-1}(t) = 0
\end{equation}
so (\ref{eq:33}) can be rewritten
\begin{equation}
  \label{eq:35}
  \begin{array}{lcl}
    \dot p_{x'}(t) = \dot p(t) \circ p^{-1}(t)   \circ   p_{x'}(t)  -    p_{x'}(t) \circ  \dot p(t) \circ p^{-1}(t)
  \end{array}
\end{equation}
The scalar part of $\dot p(t) \circ p^{-1}(t)$ is $(\dot p(t) \circ p^{-1}(t))_0 = p_o \dot p_0 + \vv{p}^T\vv{\dot p}$. Since $p$ is a unit quaternion, we have
\begin{equation}
  \label{eq:46}
  |p|=1 \implies \frac{d}{dt} (p^\top p) = 0 =  \dot p^\top p + p^\top \dot p =   2( p_o \dot p_0 + \vv{p}^T\vv{\dot p}).
\end{equation}
Therefore, the scalar part $(\dot p(t) \circ p^{-1}(t))_0 =0$.
The quaternion product $\dot p(t) \circ p^{-1}(t)$ and  $p_{x'}(t)$ is a product of quaternions with zero scalar part (see~\eqref{eq:36}), so we have 
\begin{equation}
  \label{eq:35}
  \begin{array}{lcl}
    \dot p_{x'}(t) = 2 \dot p(t) \circ p^{-1}(t)   \circ p_{x'}(t).
  \end{array}
\end{equation}
In terms of vector of $\RR^3$, this corresponds to
\begin{equation}
  \label{eq:4}
  \dot x'(t) = 2 \vv{ \dot p(t) \circ p^{-1}(t) } \times x'(t).
\end{equation}
Since $x'(t) = R(t) x$, we have $\dot x' = \dot R(t) x = \tilde \omega(t) R(t) x  = \tilde \omega(t) x'(t) $. Comparing \eqref{eq:35} and \eqref{eq:4}, we get
\begin{equation}
  \label{eq:37}
  \tilde \omega(t) = 2 \vv{ \dot p(t) \circ p^{-1}(t) } 
\end{equation}
or equivalently
\begin{equation}
  \dot p(t) \circ p^{-1}(t) = \frac 1 2 (0, \omega(t) )
  \label{eq:38}
\end{equation}
Finally, we can conclude that
\begin{equation}
  \label{eq:39}
  \dot p(t) =\frac 1 2 (0, \omega ) \circ p(t).
\end{equation}
Since $\omega(t)=R(t)\Omega(t)$, we have
\begin{equation}
  \label{eq:40}
  (0, \omega(t) ) = (0, R(t) \Omega(t) ) = p(t) \circ (0, \Omega(t) ) \circ \bar p(t) = p(t) \circ (0, \Omega(t) ) \circ  p^{-1}(t)
\end{equation}
and then
\begin{equation}
  \label{eq:41}
  \dot p(t) =\frac 1 2 p(t) \circ(0, \Omega(t) ) .
\end{equation}

The time derivation is compactly written
\begin{equation}
  \label{eq:3}
  \dot p = \frac  1 2 p  \circ(0, \Omega ) = \frac 1 2 [p_\circ] p_{\Omega} = \frac 1 2 \Psi(p)\Omega ,
\end{equation}
and using the matrix representation of product of  quaternion
we get
\begin{equation}
  \label{eq:27}
  \Psi(p) =  \begin{bmatrix}
    -p_1 & -p_2 & -p_3 \\
    p_0 & -p_3 & p_2 \\
    p_3 & p_0 & -p_1 \\
    -p_2 & p_1 & p_0 \\
  \end{bmatrix}
\end{equation}
The relation \eqref{eq:41} can be also inverted by writing
\begin{equation}
  \label{eq:42}
   (0, \Omega(t) ) = 2 p^{-1}(t) \circ \dot p(t)
\end{equation}
Using again  matrix representation of product of  quaternion, we get 
\begin{equation}
  \label{eq:42}
  \Omega(t)  = 2 \vv{p^{-1}(t) \circ \dot p(t)}  = 2  \begin{bmatrix}
    -p_1 & p_0 & p_3 & -p_2 \\
    -p_2 & -p_3 & p_0 & p_1 \\
    -p_3 & p_2 & -p_1  & p_0\\
  \end{bmatrix}\dot p(t) = 2 \Psi(p)^\top \dot p(t)
\end{equation}
Note that we have $\Psi^\top(p)\Psi(p)= I_{4\times 4 }$ and  $\Psi(p)\Psi^\top(p)= I_{3\times 3 }$ 
\paragraph{Computation of $T$ for unit quaternion} The operator $T(q)$ is directly obtained as
\begin{equation}
  T(q)=\frac 1 2 \label{eq:5}
  \begin{bmatrix}
    2 I_{3\times 3} & & 0_{3\times 3} & \\
    &   -p_1 & -p_2 & -p_3 \\
    0_{4\times 3}  &  p_0 & -p_3 & p_2 \\
    & p_3 & p_0 & -p_1 \\
    & -p_2 & p_1 & p_0 
  \end{bmatrix}
\end{equation}

\paragraph{Newton-Euler equation in compact form}

%
The Newton-Euler equation in compact form may be written as:
\begin{equation}
\label{eq:Newton-Euler-compact}
\boxed{ \left \{ 
 \begin{aligned}
  &\dot q=T(q)v, \\
  & M \dot v = F(t, q, v)
 \end{aligned}
 \right.}
\end{equation}
where $M\in\RR^{6\times6}$ is the total inertia matrix
\begin{equation}
  M:= \begin{pmatrix}
    m I_{3\times 3} & 0 \\
    0 & I 
  \end{pmatrix},
\end{equation}
and $F(t, q, v)\in \RR^6$ collects all the forces and torques applied to the body
\begin{equation}
  F(t,q,v):= \begin{pmatrix}
    f(t,x_{\cg},  v_{\cg}, R, \Omega ) \\
    I \Omega \times \Omega + M(t,x_{\cg}, v_{\cg}, R, \Omega )
  \end{pmatrix}.
\end{equation}
When a collection of bodies is considered, we will use the same notation as in~(\ref{eq:Newton-Euler-compact}) extending the definition of the variables $q,v$ and the operators $M,F$ in a straightforward way.


\begin{ndrva}
  todo :
  \begin{itemize}
  \item computation of the directional derivative of $R(\Omega)= exp(\tilde \Omega)$ in the direction $\tilde\Omega$, to get $T(\Omega)$  
  \end{itemize}
\end{ndrva}

\paragraph{Quaternion representation}If the Lie group is described by unit quaternion, we get
\begin{equation}
  \label{eq:47}
  SO(3) = \{p = (p_0,\vv{p}) \in \RR^{4}\mid |p|=1  \}
\end{equation}
with the composition law  $p_1\circ p_2$ given by the quaternion product.



Note that the concept of exponential map for Lie group that are not parameterized by matrices is also possible.


\subsection{Mechanical systems  with bilateral and unilateral constraints}
\label{section22}


Let us consider that the system~(\ref{eq:Newton-Euler-compact}) is  subjected to $m$ constraints, with $m_{e}$ holonomic bilateral 
constraints
\begin{equation}
  \label{eq:bilateral-constraints}
  h^\alpha(q)=0, \alpha \in \mathcal{E}\subset\NN,  |\mathcal E| = m_e,
\end{equation}
and  $m_{i}$ unilateral constraints
\begin{equation}
  \label{eq:unilateral-constraints}
  g_{\n}^\alpha(q)\geq 0, \alpha \in \mathcal{I}\subset\NN,  |\mathcal I| = m_i.
\end{equation} 
%
Let us denote as $J^\alpha_h(q) = \nabla^\top_q h^\alpha(q)  $ the Jacobian matrix of the bilateral constraint $h^\alpha(q)$ with respect to $q$ and as $J^\alpha_{g_\n}(q)$ respectively for $g_{\n}^\alpha(q)$  .
%
The bilateral constraints at the velocity level can be obtained as:
\begin{equation}
  \label{eq:bilateral-constraints-velocity}
 0 = \dot h^\alpha(q)= J^\alpha_h(q)\dot q = J^\alpha_h(q) T(q) v \coloneqq H^\alpha(q)  v,\quad  \alpha \in \mathcal{E}.
\end{equation}
By duality and introducing a Lagrange multiplier $\lambda^\alpha, \alpha \in \mathcal E$, the constraint generates a force applied to the body equal to $H^{\alpha,\top}(q)\lambda^\alpha$. For the unilateral constraints, a Lagrange multiplier $\lambda_{\n}^\alpha, \alpha \in \mathcal I$ is also associated and the constraints at the velocity level can also be derived as
\begin{equation}
  \label{eq:unilateral-constraints-velocity}
 0 \leq  \dot g_\n^\alpha(q)= J^\alpha_{g_\n}(q) \dot q = J^\alpha_{g_\n}(q)  T(q) v , \text{ if } g_{\n}^\alpha(q) = 0,\quad  \alpha \in \mathcal{I}. 
\end{equation}
Again, the force applied to the body is given by $(J^\alpha_{g_\n}(q) T(q))^\top\lambda^\alpha_\n$. {Nevertheless, there is no reason that $\lambda^\alpha_\n =r^\alpha_\n$ and $u_\n = J^\alpha_{g_\n}(q) T(q) v$ if the $g_n$ is not chosen as the signed distance (the gap function)}. This is the reason why  we prefer  directly define the normal and the tangential local relative velocity with respect to the {twist vector} as
\begin{equation}
  \label{eq:unilateral-constraints-velocity-kinematic1}
   u^\alpha_\n  \coloneqq G_\n^\alpha(q) v, \quad u^\alpha_\t  \coloneqq G_\t^\alpha(q) v, \quad \alpha \in \mathcal{I},
\end{equation}
and the associated force as $G_\n^{\alpha,\top}(q) r^{\alpha}_\n $ and $G_\t^{\alpha,\top}(q) r^{\alpha}_\t$. For the sake of simplicity, we use the notation $u^\alpha  \coloneqq G^\alpha(q) v$ and its associated total force generated by the contact $\alpha$ as $G^{\alpha,\top}(q) r^{\alpha} \coloneqq G_\n^{\alpha,\top}(q) r^{\alpha}_\n + G_\t^{\alpha,\top}(q) r^{\alpha}_\t $.

The complete system of equation of motion can finally be written as
\begin{numcases}{ }
  ~~\dot q = T(q)v ,\nonumber \\[0.5ex]
  ~~ M \dot v  = F(t,q,v) + H^\top(q) \lambda +  G^\top(q) r, \nonumber \\ [0.5ex]
  ~~\begin{array}{ll}
    H^\alpha(q) v  =  0 ,& \alpha \in \mathcal E \\[1ex]
    \left. \begin{array}{ll}
      r^\alpha= 0 , &\text{ if } g_{\n}^\alpha(q) > 0,\\[1ex]
      {K}^{\alpha,*} \ni \widehat u^\alpha  \bot~ r^\alpha \in {K}^\alpha, &\text{ if } g_{\n}^\alpha(q) = 0, \\[1ex]
      u_{\n}^{\alpha,+} = -e_r^\alpha u_{\n}^{\alpha,-}, &\text{ if } g_{\n}^\alpha(q) = 0 \text{ and } u_{\n}^{\alpha,-} \leq 0, 
    \end{array}\right\} & \alpha \in \mathcal I  \label{eq:NewtonEuler-uni}
\end{array}
\end{numcases}
where the definition of the variables $\lambda\in \RR^{m_e}, r\in \RR^{3m_i}$ and the operators $H,G$ are extended to collect all the variables for each constraints.

Note that all the constraints are written at the velocity integrators. {Another strong advantage is the straightforward introduction of  the contact dissipation processes that are naturally written at the velocity level such as the Newton impact law and the Coulomb friction. Indeed, in Mechanics, dissipation processes are always given in terms of rates of changes, or if we prefer, in terms of velocities.}

\paragraph{Siconos Notation} In the siconos notation, we have for the applied torques on the system the following decomposition
\begin{equation}
  F(t,q,v):= \begin{pmatrix}
    f(t,x_{\cg},  v_{\cg}, R, \Omega ) \\
    I \Omega \times \Omega + M(t,x_{\cg}, v_{\cg}, R, \Omega )
  \end{pmatrix}
  := \begin{pmatrix}
    f_{ext}(t)  - f_{int}(x_{\cg},  v_{\cg}, R, \Omega ) \\
    - M_{gyr}(\Omega) + M_{ext}(t) -  M_{int}(x_{\cg}, v_{\cg}, R, \Omega )
  \end{pmatrix}.
\end{equation}
with
\begin{equation}
  M_{gyr} := \begin{pmatrix}
     \Omega \times I\Omega
  \end{pmatrix}
\end{equation}



In the siconos notation, we have for the relation
\begin{equation}
  \label{eq:2}
   C =   J^\alpha(q) \quad CT = J^\alpha(q)T(q)
\end{equation}







\section{Time integration scheme in scheme}


\subsection{Moreau--Jean scheme based on a  $\theta$-method}
The complete Moreau--Jean scheme based on a  $\theta$-method is written as follows
 \begin{equation}
    \label{eq:Moreau--Jean-theta}
    \begin{cases}
      ~~\begin{array}{l}
        q_{k+1} = q_{k} + h T(q_{k+\theta}) v_{k+\theta} \quad \\[1ex]
        M(v_{k+1}-v_k) - h  F(t_{k+\theta}, q_{k+\theta},v_{k+\theta}) =  H^\top(q_{k+1}) Q_{k+1} + G^\top(q_{k+1}) P_{k+1},\quad\,\\[1ex]
      \end{array}\\
      ~~\begin{array}{lcl}
        \begin{array}{l}
          H^\alpha(q_{k+1}) v_{k+1}  =  0\\
        \end{array} & \left. \begin{array}{l}
          \vphantom{H^\alpha(q_{k+1}) v_{k+1}  =  0}\\[1ex]
        \end{array}\right\}    &\alpha \in \mathcal E  \\[1ex]
      ~~~P_{k+1}^\alpha= 0, &
      \left. \begin{array}{l}
          \vphantom{P_{k+1}^\alpha= 0,  \delta^\alpha_{k+1}=0}\\[1ex]
        \end{array}\right\}   & \alpha \not\in \mathcal I^\nu \\[1ex]
      % 
      % 
      \begin{array}{l}
          {K}^{\alpha,*} \ni \widehat u_{k+1}^\alpha~ \bot~ P_{k+1}^\alpha \in {K}^\alpha \\
      \end{array} &
      \left.\begin{array}{l}
          \vphantom{{K}^{\alpha,*} \ni \widehat u_{k+1}^\alpha~ \bot~ P_{k+1}^\alpha \in {K}^\alpha} \\
        \end{array}\right\}
      &\alpha \in \mathcal I^\nu\\
  \end{array}
\end{cases}
\end{equation}
where $\mathcal I^\nu$ is the set of forecast constraints, that may be evaluated as
\begin{equation}
  \label{eq:10}
  \mathcal I^\nu = \{\alpha \mid \bar g_\n^\alpha \coloneqq g_\n + \frac h 2 u^\alpha_\n \leq 0\}.
\end{equation}


\subsection{Semi-explicit version Moreau--Jean scheme based on a  $\theta$-method}

\begin{equation}
    \label{eq:Moreau--Jean-explicit}
    \begin{cases}
      ~~\begin{array}{l}
        q_{k+1} = q_{k} + h T(q_{k}) v_{k+\theta} \quad \\[1ex]
        M(v_{k+1}-v_k) - h  F(t_{k}, q_{k},v_{k}) =  H^\top(q_{k}) Q_{k+1}+  G^\top(q_{k}) P_{k+1},\quad\,\\[1ex]
      \end{array}\\
      ~~\begin{array}{lcl}
        \begin{array}{l}
          H^\alpha(q_{k+1}) v_{k+1}  =  0\\
        \end{array} & \left. \begin{array}{l}
          \vphantom{H^\alpha(q_{k+1}) v_{k+1}  =  0}\\[1ex]
        \end{array}\right\}    &\alpha \in \mathcal E  \\[1ex]
      ~~P_{k+1}^\alpha= 0, &
      \left. \begin{array}{l}
          \vphantom{P_{k+1}^\alpha= 0,  \delta^\alpha_{k+1}=0}\\[1ex]
        \end{array}\right\}   & \alpha \not\in \mathcal I^\nu \\[1ex]
      % 
      % 
      \begin{array}{l}
          {K}^{\alpha,*} \ni \widehat u_{k+1}^\alpha~ \bot~ P_{k+1}^\alpha \in {K}^\alpha \\
      \end{array} &
      \left.\begin{array}{l}
          \vphantom{{K}^{\alpha,*} \ni \widehat u_{k+1}^\alpha~ \bot~ P_{k+1}^\alpha \in {K}^\alpha} \\
        \end{array}\right\}
      &\alpha \in \mathcal I^\nu\\
  \end{array}
\end{cases}
\end{equation}

In this version, the new velocity $v_{k+1}$ can be computed explicitly, assuming that the inverse of $M$ is easily written, as

\begin{equation}
  \label{eq:Moreau--Jean-theta--explicit-v}
  v_{k+1}   =  v_k + M^{-1} h  F(t_{k}, q_{k},v_{k}) +  M^{-1} (H^\top(q_{k}) Q_{k+1}+  G^\top(q_{k}) P_{k+1})
\end{equation}


\subsection{Nearly implicit version Moreau--Jean scheme based on a  $\theta$-method implemented in siconos}

A first simplification is made considering a given value of $q_{k+1}$ in $T()$, $H()$ and $G()$ denoted by $\bar q_k$. This limits the computation of the Jacobians of this operators with respect to $q$. 
\begin{equation}
    \label{eq:Moreau--Jean-theta-nearly}
    \begin{cases}
      ~~\begin{array}{l}
        q_{k+1} = q_{k} + h T(\bar q_k) v_{k+\theta} \quad \\[1ex]
        M(v_{k+1}-v_k) - h  \theta F(t_{k+1}, q_{k+1},v_{k+1}) - h (1- \theta) F(t_{k}, q_{k},v_{k})  =  H^\top(\bar q_k) Q_{k+1} + G^\top(\bar q_k) P_{k+1},\quad\,\\[1ex]
      \end{array}\\
      ~~\begin{array}{lcl}
        \begin{array}{l}
          H^\alpha(\bar q_k) v_{k+1}  =  0\\
        \end{array} & \left. \begin{array}{l}
          \vphantom{H^\alpha(q_{k+1}) v_{k+1}  =  0}\\[1ex]
        \end{array}\right\}    &\alpha \in \mathcal E  \\[1ex]
      ~~P_{k+1}^\alpha= 0, &
      \left. \begin{array}{l}
          \vphantom{P_{k+1}^\alpha= 0,  \delta^\alpha_{k+1}=0}\\[1ex]
        \end{array}\right\}   & \alpha \not\in \mathcal I^\nu \\[1ex]
      % 
      % 
      \begin{array}{l}
          {K}^{\alpha,*} \ni \widehat u_{k+1}^\alpha~ \bot~ P_{k+1}^\alpha \in {K}^\alpha \\
      \end{array} &
      \left.\begin{array}{l}
          \vphantom{{K}^{\alpha,*} \ni \widehat u_{k+1}^\alpha~ \bot~ P_{k+1}^\alpha \in {K}^\alpha} \\
        \end{array}\right\}
      &\alpha \in \mathcal I^\nu\\
  \end{array}
\end{cases}
\end{equation}
The nonlinear residu is defined as
\begin{equation}
  \label{eq:Moreau--Jean-theta--nearly-residu}
  \mathcal R(v) =  M(v-v_k) - h  \theta F(t_{k+1}, q(v),v) - h (1- \theta) F(t_{k}, q_{k},v_{k}) - H^\top(\bar q_k) Q_{k+1} - G^\top(\bar q_k) P_{k+1}
\end{equation}
with
\begin{equation}
  \label{eq:Moreau--Jean-theta--nearly-residu1}
  q(v) = q_{k} + h T(\bar q_k)) ((1-\theta) v_k + \theta v).
\end{equation}
At each time step, we have to solve
\begin{equation}
  \label{eq:Moreau--Jean-theta--nearly-residu2}
  \mathcal R(v_{k+1}) =  0
\end{equation}
together with the constraints.

Let us write a linearization of the problem to design a Newton procedure:
\begin{equation}
  \label{eq:Moreau--Jean-theta--nearly-residu3}
  \nabla^\top_v \mathcal R(v^{\tau}_{k+1})(v^{\tau+1}_{k+1}-v^{\tau}_{k+1}) = -  \mathcal R(v^{\tau}_{k+1}).
\end{equation}
The computation of $ \nabla^\top_v \mathcal R(v^{\tau}_{k+1})$ is as follows
\begin{equation}
  \label{eq:1}
  \nabla^\top_v \mathcal R(v) = M - h \theta \nabla_v F(t_{k+1}, q(v),v)
\end{equation}
with
\begin{equation}
  \label{eq:6}
  \begin{array}{lcl}
    \nabla_v F(t_{k+1}, q(v),v) &=& D_2 F(t_{k+1}, q(v),v) \nabla_v q(v) + D_3 F(t_{k+1}, q(v),v) \\
                                &=& h \theta D_2 F(t_{k+1}, q(v),v) T(\bar q_k) + D_3 F(t_{k+1}, q(v),v) \\
  \end{array}
\end{equation}
where $D_i$ denotes the derivation with respect the $i^{th}$ variable. The complete Jacobian is then given by
\begin{equation}
  \label{eq:7}
  \nabla^\top_v \mathcal R(v) = M - h \theta D_3 F(t_{k+1}, q(v),v) - h^2 \theta^2 D_2 F(t_{k+1}, q(v),v) T(\bar q_k)
\end{equation}
In siconos, we ask the user to provide the functions $D_3 F(t_{k+1}, q ,v )$ and $D_2 F(t_{k+1}, q,v)$.

Let us denote by $W^{\tau}$ the inverse of  Jacobian of the residu,
\begin{equation}
  \label{eq:11}
  W^{\tau} = (M - h \theta D_3 F(t_{k+1}, q(v),v) - h^2 \theta^2 D_2 F(t_{k+1}, q(v),v) T(\bar q_k))^{-1}.
\end{equation}
and by $\mathcal R_{free}(v)$ the free residu,
\begin{equation}
  \label{eq:12}
  \mathcal R_{free}(v) =  M(v-v_k) - h  \theta F(t_{k+1}, q(v),v) - h (1- \theta) F(t_{k}, q_{k},v_{k}).
\end{equation}

The linear equation \ref{eq:Moreau--Jean-theta--nearly-residu3} that we have to solve is equivalent to
\begin{equation}
  \label{eq:13}
  \boxed{v^{\tau+1}_{k+1} = v^{\tau}_{k+1} - W  \mathcal R_{free}(v^\tau_{k+1}) + W   H^\top(\bar q_k) Q^{\tau+1}_{k+1} + W G^\top(\bar q_k) P^{\tau+1}_{k+1}}
\end{equation}
We define  $v_{free}$ as
\begin{equation}
  \label{eq:15}
  v_{free}  = v^{\tau}_{k+1} - W  \mathcal R_{free}(v^\tau_{k+1})
\end{equation}

The local velocity at contact can be written
\begin{equation}
  \label{eq:14}
  u^{\tau+1}_{\n,k+1} = G(\bar q_k) [  v_{free}^{\tau} + W   H^\top(\bar q_k) Q^{\tau+1}_{k+1} + W G^\top(\bar q_k) P^{\tau+1}_{k+1}]
\end{equation}
and for the equality constraints
\begin{equation}
  \label{eq:14}
  u^{\tau+1}_{k+1} = H(\bar q_k) [  v_{free}^{\tau} + W   H^\top(\bar q_k) Q^{\tau+1}_{k+1} + W G^\top(\bar q_k) P^{\tau+1}_{k+1}]
\end{equation}
Finally, we get a linear relation between $u^{\tau+1}_{\n,k+1}$ and the multiplier 
\begin{equation}
  \label{eq:16}
 \boxed{ u^{\tau+1}_{k+1} =
  \begin{bmatrix}
    H(\bar q_k) \\
    G(\bar q_k)
  \end{bmatrix} v_{free}^{\tau}
  +
  \begin{bmatrix}
    H(\bar q_k)W   H^\top(\bar q_k) & H(\bar q_k)W   G^\top(\bar q_k) \\
    G(\bar q_k)W   H^\top(\bar q_k) & G(\bar q_k)W   G^\top(\bar q_k) \\
  \end{bmatrix}
  \begin{bmatrix}
    Q^{\tau+1}_{k+1} \\
    P^{\tau+1}_{k+1}
  \end{bmatrix}}
\end{equation}






\paragraph{choices for $\bar q_k$} Two choices are possible for $\bar q_k$
\begin{enumerate}
\item $\bar q_k = q_k$
\item $\bar q_k = q^{\tau}_{k+1}$
\end{enumerate}

\begin{ndrva}

  todo list:
  
  \begin{itemize}


  \item add the projection step for the unit quaternion

  \item describe the computation of H and G that can be hybrid

    
\end{itemize}

\end{ndrva}


\subsection{Computation of the Jacobian in special case}

\paragraph{Moment of gyroscopic forces}
Let us denote by the basis vector $e_i$ given the $i^{th}$ column of the identity matrix $I_{3\times3}$. The Jacobian of $M_{gyr}$ is given by
\begin{equation}
  \label{eq:8}
  \nabla^\top_\Omega M_{gyr}(\Omega) = \nabla^\top_\Omega (\Omega \times I \Omega) =
  \begin{bmatrix}
    e_i \times I \Omega + \Omega \times I e_i, i =1,2,3
  \end{bmatrix}
\end{equation}

\paragraph{Linear internal wrench}
If the internal wrench  is given by
\begin{equation}
  \label{eq:9}
  F_{int}(t,q,v) =
  \begin{bmatrix}
    f_{int}(t,q,v)\\
    M_{int}(t,q,v)
  \end{bmatrix}
  = C v + K q, \quad C \in \RR^{6\times 6}, \quad K \in \RR^{6\times 7 }
\end{equation}
we get
\begin{equation}
  \label{eq:6}
  \begin{array}{lcl}
    \nabla_v F(t_{k+1}, q(v),v)  &=& h \theta K T(\bar q_k) + C \\
    \nabla^\top_v \mathcal R(v) &=& M - h \theta C - h^2 \theta^2 K T(\bar q_k)
  \end{array}
\end{equation}

\paragraph{External moment given in the inertial frame}

If the external moment denoted by $m_{ext} (t)$ is expressed in inertial frame, we have
\begin{equation}
  \label{eq:18}
  M_{ext}(q,t) = R^T m_{ext}(t)= \Phi(p) m_{ext}(t)
\end{equation}
In that case, $  M_{ext}(q,t)$ appears as a function $q$ and we need to compute its Jacobian w.r.t $q$. This computation needs the computation of
\begin{equation}
  \label{eq:22}
  \nabla_{p} M_{ext}(q,t) = \nabla_{p} \Phi(p) m_{ext}(t) 
\end{equation}
Let us compute first
\begin{equation}
  \label{eq:23}
  \Phi(p) m_{ext}(t)  =
  \begin{bmatrix}
    (1-2 p_2^2- 2 p_3^2)m_{ext,1} + 2(p_1p_2-p_3p_0)m_{ext,2} + 2(p_1p_3+p_2p_0)m_{ext,3}\\
    2(p_1p_2+p_3p_0)m_{ext,1}  +(1-2 p_1^2- 2 p_3^2)m_{ext,2} + 2(p_2p_3-p_1p_0)m_{ext,3}\\
    2(p_1p_3-p_2p_0)m_{ext,1}  + 2(p_2p_3+p_1p_0)m_{ext,2}  + (1-2 p_1^2- 2 p_2^2)m_{ext,3}\\
  \end{bmatrix}
\end{equation}
Then we get
\begin{equation}
  \label{eq:24}
  \begin{array}{l}
  \nabla_{p} \Phi(p) m_{ext}(t)  =\\
  \begin{bmatrix}
    -2 p_3 m_{ext,2} + 2 p_2 m_{ext,3} & 2p_2 m_{ext,2}+2 p_3 m_{ext,3}  & -4 p_2 m_{ext,1} +2p_1 m_{ext,2}+2 p_0 m_{ext,3} & -3 p_3 m_{ext,1} -2p_0 m_{ext,2} +2 p_1m_{ext,3}  \\
    2p_3 m_{ext,1} -2p_1m_{ext,3}  & 2p_2m_{ext,1} -4p_1 m_{ext,2} -2p_1 m_{ext,3} & & &  \\
  \end{bmatrix}
  \end{array}
\end{equation}





\subsection{Siconos implementation}

The expression:~$\mathcal R_{free}(v^\tau_{k+1}) = M(v-v_k) - h  \theta F(t_{k+1}, q(v^\tau_{k+1}),v^\tau_{k+1}) - h (1- \theta) F(t_{k}, q_{k},v_{k})$ is computed in {\tt MoreauJeanOSI::computeResidu()} and saved in {\tt ds->workspace(DynamicalSystem::freeresidu)}


The expression:~$\mathcal R(v^\tau_{k+1}) =\mathcal R_{free}(v^\tau_{k+1}) - h (1- \theta) F(t_{k}, q_{k},v_{k}) - H^\top(\bar q_k) Q_{k+1} - G^\top(\bar q_k) P_{k+1}  $ is computed in {\tt MoreauJeanOSI::computeResidu()} and saved in {\tt ds->workspace(DynamicalSystem::free)}.
\begin{ndrva}
  really a bad name for the buffer {\tt ds->workspace(DynamicalSystem::free)}. Why we are chosing this name ? to save some memory ?
\end{ndrva}


The expression:~$v_{free}  = v^{\tau}_{k+1} - W  \mathcal R_{free}(v^\tau_{k+1})$ is compute in {\tt MoreauJeanOSI::computeFreeState()} and saved in {\tt d->workspace(DynamicalSystem::free)}. 



The computation:~ $v^{\tau+1}_{k+1} = v_{free} + W   H^\top(\bar q_k) Q^{\tau+1}_{k+1} + W G^\top(\bar q_k) P^{\tau+1}_{k+1}$ is done in {\tt MoreauJeanOSI::updateState} and stored in {\tt d->twist()}.\\


%%% Local Variables: 
%%% mode: latex
%%% TeX-master: "DevNotes"
%%% End: 
