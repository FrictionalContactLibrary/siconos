

\subsection{Slidding ?}
It consists in finding $\alpha >0$ and $R \in \partial K_{\mu}$ such that $-\alpha \left(\begin{array}{l} 0\\ R_T\end{array}\right)=MR+q$. That is :
  \begin{equation}
\label{eq_quartic1}
\left[\begin{array}{c}
M+ \left(\begin{array}{ccc} 0&0&0\\ 0&\alpha&0 \\ 0&0&\alpha \end{array}\right)
\end{array}\right]R+q=0
\end{equation}

  \subsubsection{$R_T$ is on a conic}
  The first line of the system~\ref{eq_quartic1} and the $R \in \partial K_{\mu}$ is the intersection between a plan and a cone in $\mathbb{R}^3$, endeed:
  \begin{equation}
\label{eq_quartic2}
\begin{array}{l}
 \mu R_N =  \parallel R_T \parallel  \\
\frac{M_{11}}{\mu} \parallel R_T \parallel = -q_1-M_{12}R_{T1}-M_{13}R_{T2}
\end{array}
\end{equation}
That is:
\begin{equation}
\label{eq_quartic2}
\begin{array}{l}
\mu^2 R_N^2 =  (R_{T1}^2 +R_{T1}^2)  \\
\frac{M_{11}^2}{\mu^2} (R_{T1}^2 +R_{T1}^2)=(-q_1-M_{12}R_{T1}-M_{13}R_{T2})^2
\end{array}
\end{equation}
That means that $R_T$ is contained in a conic,  focus and directrice are:
\begin{equation}
\label{eq_quartic3}
\begin{array}{l}
\mathcal{D} : q_1+M_{12}R_{T1}+M_{13}R_{T2} =0  \\
focus : \mathcal{O}\\
\frac{M_{11}^2}{\mu^2}  Dist(\mathcal{O}, R_T) ^2=Dist(\mathcal{D},R_T)^2 (M_{12}^2+M_{13}^2)\\
\frac{Dist(\mathcal{O}, R_T)}{Dist(\mathcal{D},R_T)}=\frac{\mu\sqrt{(M_{12}^2+M_{13}^2)}}{M_{11} }=e
\end{array}
\end{equation}
The parametric equation is:
\begin{equation}
\label{eq_quartic4}
\begin{array}{l}
R_{T1}=r cos(\theta )\\
R_{T2}=r sin(\theta )\\
r=\frac{p}{1+ecos(\theta - \phi)}
\end{array}
\end{equation}
With $p$ an simple expression of $M_{11},M_{12},M_{13}$, and $\phi$ a constant angle between $\mathcal{D}$ and $(O,R_{T1})$
\subsubsection{The two last line of the system~\ref{eq_quartic1}}
\begin{equation}
\label{eq_quartic5}
\frac{\parallel R_T \parallel}{\mu} \tilde M_{1.} +\left(\tilde M+\left(\begin{array}{cc} \alpha&0 \\ 0&\alpha \end{array}\right)\right)R_T+\tilde q=0
\end{equation}
$\tilde M$ is symetric, so it exists a unitary matrix $V$ such that $V \tilde M V^T = \left(\begin{array}{cc} d_1&0 \\ 0&d_2 \end{array}\right)$.  One can get:
\begin{equation}
\label{eq_quartic6}
\frac{\parallel R_T \parallel}{\mu} V \tilde M_{1.} +V \left(\tilde M+\left(\begin{array}{cc} \alpha&0 \\ 0&\alpha \end{array}\right)\right)V^TVR_T+V\tilde q=0
\end{equation}
Rename:
\begin{equation}
\label{eq_quartic7}
  \frac{\parallel \bar R_T \parallel}{\mu} \bar M_{1.} +\left(\begin{array}{cc} d_1+\alpha&0 \\ 0&d_2+\alpha \end{array}\right)\overline R_T+\bar q=0
  \end{equation}
In the plan, either $V$ is a rotation or a symetrie. So $ \bar R_T=VR_T$ is a conic with the same focus and a rotated directrice, it means that it exists $\phi_1$ such that :

\begin{equation}
\label{eq_quartic8}
\begin{array}{l}
\bar R_{T1}=r cos(\theta )\\
\bar R_{T2}=r sin(\theta )\\
r=\frac{p}{1+ecos(\theta - \phi_1)}
\end{array}
\end{equation}
The equation~\ref{eq_quartic7} is :
\begin{equation}
\label{eq_quartic9}
\begin{array}{l}
  (d_1+\alpha)\bar R_{T1}=-\bar q_1+a_1 \parallel R_T \parallel\\
(d_2+\alpha)\bar R_{T2}=-\bar q_2+a_2 \parallel R_T \parallel
\end{array}
\end{equation}
The case ($\bar R_{T1} = 0$ or  $\bar R_{T2} = 0$) has to be examine. We try to eliminate $alpha$:
\begin{equation}
\label{eq_quartic10}
  \begin{array}{l}
    d_1 \bar R_{T1} \bar R_{T2}+\alpha \bar R_{T1} \bar R_{T2} =-\bar q_1\bar R_{T2}+a_1 \bar R_{T2} \parallel R_T \parallel\\
d_2 \bar R_{T1} \bar R_{T2}+\alpha \bar R_{T1} \bar R_{T2} =-\bar q_2\bar R_{T1}+a_2 \bar R_{T1} \parallel R_T \parallel
\end{array}
\end{equation}
that leads to:
\begin{equation}
\label{eq_quartic10}
  (d_1-d_2) \bar R_{T1} \bar R_{T2}=-\bar q_1\bar R_{T2}+\bar q_2\bar R_{T1}+(a_1 \bar R_{T2}-a_2 \bar R_{T1}) \parallel R_T \parallel\\
\end{equation}
The parametric expression of $\bar R_T$ leads to:
\begin{equation}
\label{eq_quartic11}
\begin{array}{l}
  (d_1-d_2)r^2cos(\theta )sin(\theta )=-\bar q_1rsin(\theta )+\bar q_2rcos(\theta )+r(a_1 rsin(\theta )-a_2 rcos(\theta )) \\
  \textrm{ie:}(d_1-d_2)rcos(\theta )sin(\theta )=-\bar q_1sin(\theta )+\bar q_2cos(\theta )+r(a_1 sin(\theta )-a_2 cos(\theta ))\\
  \end{array}
\end{equation}
with the expression of r:
\begin{equation}
\label{eq_quartic12}
\begin{array}{l}
(d_1-d_2)\frac{p}{1+ecos(\theta - \phi_1)}cos(\theta )sin(\theta )=\\-\bar q_1sin(\theta )+\bar q_2cos(\theta )+\frac{p}{1+ecos(\theta - \phi_1)}(a_1  sin(\theta )-a_2 cos(\theta ))\\\\
\textrm{ie:}(d_1-d_2)pcos(\theta )sin(\theta )=\\(1+ecos(\theta - \phi_1))(-\bar q_1sin(\theta )+\bar q_2cos(\theta ))+p(a_1  sin(\theta )-a_2 cos(\theta ))\\\\
\textrm{ie:}(d_1-d_2)pcos(\theta )sin(\theta )=\\(1+e(cos(\theta)cos(\phi_1)+sin(\theta)sin(\phi_1)))(-\bar q_1sin(\theta )+\bar q_2cos(\theta ))+p(a_1  sin(\theta )-a_2 cos(\theta ))\\\\
\textrm{ie:}(d_1-d_2)pcos(\theta )sin(\theta )+\\(1+ecos(\theta)cos(\phi_1)+esin(\theta)sin(\phi_1))(\bar q_1sin(\theta )-\bar q_2cos(\theta ))+p(-a_1  sin(\theta )+a_2 cos(\theta ))=0
 \end{array}
\end{equation}
rename :
\begin{equation}
\label{eq_quartic13}
\begin{array}{l}
Acos(\theta )^2+Bsin(\theta)^2+Csin(\theta )cos(\theta )+Dsin(\theta )+Ecos(\theta )=0
 \end{array}
\end{equation}
with
\begin{equation}
\label{eq_quartic12}
\begin{array}{l}
A=- e\bar q_2cos(\phi_1)\\
B=e \bar q_1sin(\phi_1)\\
C=(d_1-d_2)p+ecos(\phi_1)\bar q_1-esin(\phi_1)\bar q_2\\
D=\bar q_1-pa_1\\
E=-\bar q_2+pa_2\\
\end{array}
\end{equation}
rename :
Using the following set of unknown :
\begin{equation}
\label{eq_quartic14}
\begin{array}{l}
t=tan(\theta /2)\\
sin(\theta )=\frac{2t}{1+t^2}\\
cos(\theta )=\frac{1-t^2}{1+t^2}
 \end{array}
\end{equation}
leads to:
\begin{equation}
\label{eq_quartic13}
\begin{array}{l}
  A\frac{(1-t^2)^2}{1+t^2} +B\frac{4t^2}{1+t^2}+ C\frac{2t(1-t^2)}{1+t^2}+D2t+E(1-t^2)=0\\
\textrm{ie:}A(1-t^2)^2 + 4Bt^2+C2t(1-t^2)+2Dt(1+t^2)+E(1-t^2)(1+t^2)=0\\\\
\textrm{ie:}P_4=A-E\qquad P_3=-2C+2D \qquad P_2=4B-2A \qquad P_1=2C+2D \qquad P_0=A+E
 \end{array}
\end{equation}
Finally, we get 4 possible values for $R_T$, checking the sign of $\alpha$ and $R_N$ selects the solutions.

\subsubsection{case $R_{T12}=0$}
From~\ref{eq_quartic9}, $R_{T1}$ leads to:
\begin{equation}
\label{eq_quartic14}
\begin{array}{l}
  \parallel R_T \parallel=|\bar R_{T2}|=\frac{\bar q_1}{a_1}\\\\
  \bar R_T=\left(\begin{array}{c} 0 \\ \pm \frac{\bar q_1}{a_1} \end{array}\right)
 \end{array}
\end{equation}

From~\ref{eq_quartic9}, $R_{T2}$ leads to:
\begin{equation}
\label{eq_quartic14}
\begin{array}{l}
  \parallel R_T \parallel=|\bar R_{T1}|=\frac{\bar q_2}{a_2}\\\\
  \bar R_T=\left(\begin{array}{c}  \pm \frac{\bar q_2}{a_2} \\ 0 \end{array}\right)
 \end{array}
\end{equation}

From $\bar R_T$, we have to check the coherence with the equation~\ref{eq_quartic8}. If it is on the conic,  we compute R, and the sign condition of the equation~\ref{eq_quartic1} must be check.
