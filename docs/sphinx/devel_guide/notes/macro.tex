%$Id: macro.tex,v 1.10 2004/12/08 13:38:58 acary Exp $


%\usepackage{a4wide}
\textheight 25cm
\textwidth 16.5cm
\topmargin -1cm
%\evensidemargin 0cm
\oddsidemargin 0cm
\evensidemargin0cm
\usepackage{layout}


\usepackage{amsmath}
\usepackage{amssymb}
\usepackage{minitoc}
%\usepackage{glosstex}
\usepackage{colortbl}
\usepackage{hhline}
\usepackage{longtable}

%\usepackage{glosstex}
%\def\glossaryname{Glossary of Notation}
\def\listacronymname{Acronyms}

\usepackage[outerbars]{changebar}\setcounter{changebargrey}{20}
%\glxitemorderdefault{acr}{l}

%\usepackage{color}
\usepackage{graphicx,epsfig}
\graphicspath{{./Figures/}}
\usepackage[T1]{fontenc}
\usepackage{rotating}

%\usepackage{algorithmic}
%\usepackage{algorithm}
\usepackage{ntheorem}
\usepackage{natbib}


%\renewcommand{\baselinestretch}{2.0}
\setcounter{tocdepth}{2}     % Dans la table des matieres
\setcounter{secnumdepth}{3}  % Avec un numero.



\newtheorem{definition}{Definition}
\newtheorem{lemma}{Lemma}
\newtheorem{claim}{Claim}
\newtheorem{remark}{Remark}
\newtheorem{assumption}{Assumption}
\newtheorem{example}{Example}
\newtheorem{conjecture}{Conjecture}
\newtheorem{corollary}{Corollary}
\newtheorem{OP}{OP}
\newtheorem{problem}{Problem}
\newtheorem{theorem}{Theorem}


\newcommand{\CC}{\mbox{\rm $~\vrule height6.6pt width0.5pt depth0.25pt\!\!$C}}
\newcommand{\ZZ}{\mbox{\rm \lower0.3pt\hbox{$\angle\!\!\!$}Z}}
\newcommand{\RR}{\mbox{\rm $I\!\!R$}}
\newcommand{\HH}{\mbox{\rm $I\!\!H$}}
\newcommand{\NN}{\mbox{\rm $I\!\!N$}}

\newcommand{\Mnn}{\mathcal M^{n\times n}}
\newcommand{\Mnp}[2]{\ensuremath{\mathcal M^{#1\times #2}}}



\newcommand{\Frac}[2]{\displaystyle \frac{#1}{#2}}

\newcommand{\DP}[2]{\displaystyle \frac{\partial {#1}}{\partial {#2}}}

% c++ variables writting
\newcommand{\varcpp}[1]{\textit{#1}}
% itemize
\newcommand{\bei}{\begin{itemize}}
\newcommand{\ei}{\end{itemize}}

\newcommand{\ie}{i.e.}
\newcommand{\eg}{e.g.}
\newcommand{\cf}{c.f.}
\newcommand{\putidx}[1]{\index{#1}\textit{#1}}

\def\Er{{\rm I\! R}}
\def\En{{\rm I\! N}} 
\def\Ec{{\rm I\! C}}
 
\def\zc{\hat{z}}
\def\wc{\hat{w}}

\font\tete=cmr8 at 8 pt


% normal tangent
\def\n{{\hbox{\tiny{N}}}}
\def\t{{\hbox{\tiny{T}}}}
\def\nt{\hbox{\tiny{NT}}}
\def\nsf{\hbox{\tiny{\textsf N}}}
\def\tsf{\hbox{\tiny{\textsf T}}}
\def\sigman{\sigma_{\n}}
\def\sigmat{\sigma_{\t}}
\def\sigmant{\sigma_{\nt}}
\def\epsn{\epsilon_{\n}}
\def\epst{\epsilon_{\t}}
\def\epsnt{\epsilon_{\nt}}
\def\eps{\epsilon}
\def\veps{\varepsilon}
\def\sig{\sigma}
\def\Rn{R_{\n}}
\def\Rt{R_{\t}}
\def\cn{c_{\n}}
\def\Cn{C_{\n}}
\def\ct{c_{\t}}
\def\Ct{C_{\t}}
\def\un{u_{\n}}
\def\ut{\buu_{\t}}
\def\uut{u_{\t}}
\def\unc{u_{\n}^c}
\def\utc{\buu_{\t}^c}
\def\vn{v_{\n}}
\def\vt{v_{\t}}
\def\rr{\hbox{\tiny{\textsf R}}}
\def\irr{\hbox{\tiny{\textsf{IR}}}}
\def\rn{r_{\n}}
\def\rt{\brr_{\t}}
\def\rnc{r_{\n}^c}
\def\rtc{\brr_{\t}^c}
\def\trn{\Tilde{r}_{\n}}
\def\trt{\Tilde{\brr}_{\t}}
\def\tr{\Tilde{\brr}}
\def\tv{\Tilde{\bvv}}
\def\vn{v_{\n}}
\def\vt{\bvv_{\t}}
\def\adh{\mathsf{adh}}
\def\adj{\hbox{\tiny{\textsf{adj}}}}
\def\adjc{\hbox{\tiny{\textsf{adjC}}}}
\def\adja{\hbox{\tiny{\textsf{adjA}}}}
\def\cc{\hbox{\tiny{\textsf C}}}
\def\ca{\hbox{\tiny{\textsf A}}}

\DeclareMathOperator{\proj}{proj}
\DeclareMathOperator{\expm}{expm}
\DeclareMathOperator{\dexp}{dexp}
\DeclareMathOperator{\dlexp}{d^l exp}
\DeclareMathOperator{\drexp}{d^r exp}
\DeclareMathOperator{\dexpm}{dexpm}
\DeclareMathOperator{\expq}{expq}
\DeclareMathOperator{\dexpq}{dexpq}
\DeclareMathOperator{\Ad}{Ad}
\DeclareMathOperator{\ad}{ad}
\DeclareMathOperator{\dd}{d}



%%  Les ensembles de nombres  C. Fiorio (fiorioÊatÊmath.tu-berlin.de) 
%
\def\nbR{\ensuremath{\mathrm{I\!R}}} % IR
\def\nbN{\ensuremath{\mathrm{I\!N}}} % IN
\def\nbF{\ensuremath{\mathrm{I\!F}}} % IF
\def\nbH{\ensuremath{\mathrm{I\!H}}} % IH
\def\nbK{\ensuremath{\mathrm{I\!K}}} % IK
\def\nbL{\ensuremath{\mathrm{I\!L}}} % IL
\def\nbM{\ensuremath{\mathrm{I\!M}}} % IM
\def\nbP{\ensuremath{\mathrm{I\!P}}} % IP

%----------------------------------------------------------------------
%                  Modification des subsubsections
%----------------------------------------------------------------------
\makeatletter
\renewcommand\thesubsubsection{\thesubsection.\@alph\c@subsubsection}
\makeatother

%----------------------------------------------------------------------
%             Redaction note environnement
%----------------------------------------------------------------------
\makeatletter
\theoremheaderfont{\scshape}
\theoremstyle{marginbreak}
\theorembodyfont{\upshape}
%\newtheorem{rque}{\bf Remarque}[chapter]
%\newtheorem{rque1}{\bf \fsc{Remarque}}[chapter] !!! \fsc est une commande french
\newtheorem{ndr1}{\textbf{\textsc{Redaction note}}}[section]

\newenvironment{ndr}%
{%
\tt
%\centerline{---oOo---}
\noindent\begin{ndr1}%
}%
{%
\begin{flushright}%
%\vspace{-1.5em}\ding{111}
\end{flushright}%
\end{ndr1}%
%\centerline{---oOo---}
}

\makeatother

%----------------------------------------------------------------------
%             Redaction note environnement V.ACARY
%----------------------------------------------------------------------
\makeatletter
\theoremheaderfont{\scshape}
\theoremstyle{marginbreak}
\theorembodyfont{\upshape}
%\newtheorem{rque}{\bf Remarque}[chapter]
%\newtheorem{rque1}{\bf \fsc{Remarque}}[chapter] !!! \fsc est une commande french
\newtheorem{ndr1va}{\textbf{\textsc{Redaction note V. ACARY}}}[section]

\newenvironment{ndrva}%
{%
\tt
%\centerline{---oOo---}
\noindent\begin{ndr1va}%
}%
{%
\begin{flushright}%
%\vspace{-1.5em}\ding{111}
\end{flushright}%
\end{ndr1va}%
%\centerline{---oOo---}
}

\makeatother
%----------------------------------------------------------------------
%             Redaction note environnement V.ACARY
%----------------------------------------------------------------------
\makeatletter
\theoremheaderfont{\scshape}
\theoremstyle{marginbreak}
\theorembodyfont{\upshape}
%\newtheorem{rque}{\bf Remarque}[chapter]
%\newtheorem{rque1}{\bf \fsc{Remarque}}[chapter] !!! \fsc est une commande french
\newtheorem{ndr1fp}{\textbf{\textsc{Redaction note F. PERIGNON}}}[section]

\newenvironment{ndrfp}%
{%
\tt
%\centerline{---oOo---}
\noindent\begin{ndr1fp}%
}%
{%
\begin{flushright}%
%\vspace{-1.5em}\ding{111}
\end{flushright}%
\end{ndr1fp}%
%\centerline{---oOo---}
}

\makeatother
%----------------------------------------------------------------------
%                  Chapter head enviroment
%----------------------------------------------------------------------
\newenvironment{chapter_head}
{%
\begin{center}%
-------------------- oOo --------------------\\%
\ \\%
\begin{minipage}[]{14cm}%
\noindent\normalsize\advance\baselineskip-1pt %
}%
{%
\par\end{minipage}%
\ \\%
\ \\%
-------------------- oOo --------------------
\end{center}%
\vspace*{\stretch{1}}%
\clearpage%
\thispagestyle{empty}%
\vspace*{\stretch{1}}%
\minitoc%
\vspace*{\stretch{2}}%
\clearpage%
}


\newcommand{\contract}{{\,:\,}}

%%% Local Variables: 
%%% mode: latex
%%% TeX-master: "report"
%%% End: 
