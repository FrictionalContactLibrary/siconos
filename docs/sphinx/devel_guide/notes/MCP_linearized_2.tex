

\subsection{The special case of Newton's linearization of~(\ref{eq:toto1}) with FirstOrderType2R~(\ref{first-DS2})} 


Let us now proceed with the time discretization of~(\ref{eq:toto1}) with FirstOrderType2R~(\ref{first-DS2})  by a fully implicit scheme : 
\begin{equation}
  \begin{array}{l}
    \label{eq:mlcp2-toto1-DS2}
     M x_{k+1} = M x_{k} +h\theta f(x_{k+1},t_{k+1})+h(1-\theta) f(x_k,t_k) + h \gamma r(t_{k+1})
     + h(1-\gamma)r(t_k)  \\[2mm]
     y_{k+1} =  h(t_{k+1},x_{k+1},\lambda _{k+1}) \\[2mm]
     r_{k+1} = g(t_{k+1},\lambda_{k+1})\\[2mm]
  \end{array}
\end{equation}


 \paragraph{Newton's linearization of the first line of~(\ref{eq:mlcp2-toto1-DS2})} The linearization of the first line of the  problem~(\ref{eq:mlcp2-toto1-DS2}) is similar to the previous case so that (\ref{eq:rfree-2}) is still valid.


 \paragraph{Newton's linearization of the second  line of~(\ref{eq:mlcp2-toto1-DS2})} The linearization of the second line of the  problem~(\ref{eq:mlcp2-toto1-DS2}) is similar to the previous case so that (\ref{eq:NL11y}) is still valid.

 \paragraph{Newton's linearization of the third  line of~(\ref{eq:mlcp2-toto1-DS2})}
Since $ K^{\alpha}_{k+1} = \nabla_xg(t_{k+1},\lambda ^{\alpha}_{k+1}) = 0 $, the linearization of the third line of (\ref{eq:mlcp2-toto1-DS2}) reads as
\begin{equation}
  \label{eq:mlcp2-rrL}
  \begin{array}{l}
    \boxed{r^{\alpha+1}_{k+1} = g(t_{k+1},\lambda ^{\alpha}_{k+1})     + B^{\alpha}_{k+1} ( \lambda^{\alpha+1}-  \lambda^{\alpha}_{k+1} )}       
  \end{array}
\end{equation}


\paragraph{Reduction to a linear relation between  $x^{\alpha+1}_{k+1}$ and
$\lambda^{\alpha+1}_{k+1}$}

Inserting (\ref{eq:mlcp2-rrL}) into~(\ref{eq:rfree-11}), we get the following linear relation between $x^{\alpha+1}_{k+1}$ and
$\lambda^{\alpha+1}_{k+1}$, we get the linear relation
\begin{equation}
  \label{eq:mlcp2-rfree-13}
  \begin{array}{l}
 \boxed{   x^{\alpha+1}_{k+1}\stackrel{\Delta}{=} x^\alpha_p + \left[ h \gamma (W^{\alpha}_{k+1})^{-1}    B^{\alpha}_{k+1} \lambda^{\alpha+1}_{k+1}\right]}
   \end{array}
\end{equation}
with 
\begin{equation}
  \boxed{x^\alpha_p \stackrel{\Delta}{=}  h\gamma(W^{\alpha}_{k+1} )^{-1}\left[g(t_{k+1},\lambda^{\alpha}_{k+1}) 
    -B^{\alpha}_{k+1} (\lambda^{\alpha}_{k+1}) \right ] +x^\alpha_{\free}}
\end{equation}
and
\begin{equation}
  \label{eq:mlcp2-NL9}
  \begin{array}{l}
    W^{\alpha}_{k+1} \stackrel{\Delta}{=} M-h\theta A^{\alpha}_{k+1}\\
  \end{array}
\end{equation}


\paragraph{Reduction to a linear relation between  $y^{\alpha+1}_{k+1}$ and
$\lambda^{\alpha+1}_{k+1}$}

Inserting (\ref{eq:mlcp2-rfree-13}) into (\ref{eq:NL11y}), we get the following linear relation between $y^{\alpha+1}_{k+1}$ and $\lambda^{\alpha+1}_{k+1}$, 
\begin{equation}
   \begin{array}{l}
 y^{\alpha+1}_{k+1} = y_p + \left[ h \gamma C^{\alpha}_{k+1} ( W^{\alpha}_{k+1})^{-1}  B^{\alpha}_{k+1} + D^{\alpha}_{k+1} \right]\lambda^{\alpha+1}_{k+1}
   \end{array}
\end{equation}
with 
\begin{equation}\boxed{
y_p = y^{\alpha}_{k+1} -\mathcal R^{\alpha}_{yk+1} + C^{\alpha}_{k+1}(x_q) -
D^{\alpha}_{k+1} \lambda^{\alpha}_{k+1} }
\end{equation}
\textcolor{red}{
  \begin{equation}
    \boxed{ x^\alpha_q= x^\alpha_p - x^{\alpha}_{k+1}\label{eq:mlcp2-xqq}}
  \end{equation}
}

\subsection{The special case of Newton's linearization of~(\ref{eq:toto1}) with FirstOrderType1R~(\ref{first-DS1})} 


Let us now proceed with the time discretization of~(\ref{eq:toto1}) with FirstOrderType1R~(\ref{first-DS1})  by a fully implicit scheme : 
\begin{equation}
  \begin{array}{l}
    \label{eq:mlcp3-toto1-DS1}
     M x_{k+1} = M x_{k} +h\theta f(x_{k+1},t_{k+1})+h(1-\theta) f(x_k,t_k) + h \gamma r(t_{k+1})
     + h(1-\gamma)r(t_k)  \\[2mm]
     y_{k+1} =  h(t_{k+1},x_{k+1}) \\[2mm]
     r_{k+1} = g(t_{k+1}\lambda_{k+1})\\[2mm]
  \end{array}
\end{equation}

The previous derivation is valid with $ D^{\alpha}_{k+1} =0$.



%%% Local Variables: 
%%% mode: latex
%%% TeX-master: "DevNotes"
%%% End: