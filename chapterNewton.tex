 \begin{table}[!ht]
  \begin{tabular}{|l|l|}
    \hline
    author  & O.Bonnefon, V. Acary\\
    \hline
    date    & Sept, 07, 2007 \\ 
    last update        &Feb, 2011 \\ 
    \hline
    version &  \\
    \hline
  \end{tabular}
\end{table}



This section is devoted to the implementation and the study  of the algorithm. The interval of integration is $[0,T]$, $T>0$, and a grid $t_{0}=0$, $t_{k+1}=t_{k}+h$, $k \geq 0$, $t_{N}=T$ is constructed. The approximation of a function $f(\cdot)$ on $[0,T]$ is denoted as $f^{N}(\cdot)$, and is a piecewise constant function, constant on the intervals $[t_{k},t_{k+1})$. We denote $f^{N}(t_{k})$ as $f_{k}$. The time-step is $h>0$. 


\section{Various first order dynamical systems with input/output relations}

\paragraph{Fully nonlinear case}
Let us introduce the following system, 
\begin{equation}
\begin{array}{l}
M \dot{x}(t) = f(x(t),t) + r(t)  \\[2mm]
y(t) = h(t,x(t),\lambda (t)) \\[2mm]
r(t) = g(t,x(t),\lambda (t) ) \\[2mm]
\end{array}
\label{first-DS}
\end{equation}
where $\lambda(t) \in \RR^m$  and $y(t) \in \RR^m$ are  complementary variables related through a multi-valued mapping.   According to the class of systems, we are studying, the function $f$ and $g$ are defined by a fully nonlinear framework or by affine functions. We have decided to present the time-discretization in its full generality and specialize the algorithms for each cases in Section~\ref{Sec:Spec}. This fully nonlinear case is not  implemented in Siconos yet. This fully general case is not yet implemented in Siconos.

\paragraph{FirstOrderType2R}
Let us introduce a new notation, 
\begin{equation}
\begin{array}{l}
M \dot{x}(t) = f(x(t),t) + r(t)  \\[2mm]
y(t) = h(t,x(t),\lambda (t)) \\[2mm]
r(t) = g(t,\lambda (t) ) \\[2mm]
\end{array}
\label{first-DS2}
\end{equation}
This case is implemented in Siconos with the relation FirstOrderType2R.




\paragraph{Linear case }Let us introduce a new notation, 
\begin{equation}
\begin{array}{l}
M \dot{x}(t) = Ax(t) + r(t)  +b(t)\\[2mm]
y(t) = h(x(t),\lambda (t),z) = Cx + Fz + D \lambda  \\[2mm]
r(t) = g(t,\lambda (t) ) = B \lambda \\[2mm]
\end{array}
\label{first-DS3}
\end{equation}


\section{Time--discretizations}



\subsection{Standard $\theta-\gamma$ scheme.}
Let us now proceed with the time discretization of (\ref{first-DS3}) by a fully implicit scheme : 
\begin{equation}
  \begin{array}{l}
    \label{eq:toto1}
     M x_{k+1} = M x_{k} +h\theta f(x_{k+1},t_{k+1})+h(1-\theta) f(x_k,t_k) + h \gamma r(t_{k+1})
     + h(1-\gamma)r(t_k)  \\[2mm]
     y_{k+1} =  h(t_{k+1},x_{k+1},\lambda _{k+1}) \\[2mm]
     r_{k+1} = g(x_{k+1},\lambda_{k+1},t_{k+1})\\[2mm]
  \end{array}
\end{equation}
where $\theta = [0,1]$ and $\gamma \in [0,1]$. As in \cite{acary2008}, we call the problem \eqref{eq:toto1} the ``one--step nonsmooth problem''.



 This time-discretization is slightly more general than a standard implicit Euler scheme. The main discrepancy lies in the choice of a $\theta$-method to integrate the nonlinear term. For $\theta=0$, we retrieve the explicit integration of the smooth and  single valued term $f$. Moreover for $\gamma =0$, the term $g$ is explicitly evaluated. The flexibility in the choice of $\theta$ and $\gamma$ allows the user to improve and control the accuracy, the stability and the numerical damping of the proposed method. For instance, if the smooth dynamics given by $f$ is stiff, or if we have to use big step sizes for practical reasons, the choice of $\theta > 1/2$ offers better stability with the respect to $h$.

\subsection{Full $\theta-\gamma$ scheme}

  \begin{equation}
    \begin{array}{l}
      \label{eq:toto1-ter}
      M x_{k+1} = M x_{k} +h f(x_{k+\theta},t_{k+1}) + h r(t_{k+\gamma}) \\[2mm]
      y_{k+\gamma} =  h(t_{k+\gamma},x_{k+\gamma},\lambda _{k+\gamma}) \\[2mm]
      r_{k+\gamma} = g(x_{k+\gamma},\lambda_{k+\gamma},t_{k+\gamma})\\[2mm]
      \mbox{NsLaw} ( y_{k+\gamma} , \lambda_{k+\gamma})
    \end{array}
\end{equation}

\clearpage
\section{Newton's linearization of~(\ref{eq:toto1})} 

Due to the fact that  two of the  studied classes of systems that are studied in this paper are affine functions in terms of $f$ and $g$, we propose to solve the "one--step nonsmooth problem'' (\ref{eq:toto1}) by performing an external Newton linearization.

 \paragraph{Newton's linearization of the first line of~(\ref{eq:toto1})} The first line of the  problem~(\ref{eq:toto1}) can be written under the form of a residue $\mathcal R$ depending only on $x_{k+1}$ and $r_{k+1}$ such that 
\begin{equation}
  \label{eq:NL3}
  \mathcal R (x_{k+1},r _{k+1}) =0
\end{equation}
with $\mathcal R(x,r) = M(x - x_{k}) -h\theta f( x , t_{k+1}) - h(1-\theta)f(x_k,t_k) - h\gamma r
- h(1-\gamma)r_k$.
The solution of this system of nonlinear equations is sought as a limit of the sequence $\{ x^{\alpha}_{k+1},r^{\alpha}_{k+1} \}_{\alpha \in \NN}$ such that
 \begin{equation}
   \label{eq:NL7}
   \begin{cases}
     x^{0}_{k+1} = x_k \\ \\
     \mathcal R_L( x^{\alpha+1}_{k+1},r^{\alpha+1}_{k+1}) = \mathcal
     R(x^{\alpha}_{k+1},r^{\alpha}_{k+1})  + \left[ \nabla_{x} \mathcal
     R(x^{\alpha}_{k+1},r^{\alpha}_{k+1})\right] (x^{\alpha+1}_{k+1}-x^{\alpha}_{k+1} ) +
     \left[ \nabla_{r} \mathcal R(x^{\alpha}_{k+1},r^{\alpha}_{k+1})\right] (r^{\alpha+1}_{k+1} - r^{\alpha}_{k+1} ) =0
 \end{cases}
\end{equation}
\begin{ndrva}
  What about $r^0_{k+1}$ ?
\end{ndrva}

The residu free is also defined (useful for implementation only):
\[\mathcal R _{free}(x) \stackrel{\Delta}{=}  M(x - x_{k}) -h\theta f( x , t_{k+1}) - h(1-\theta)f(x_k,t_k),\]
which yields
\[\mathcal R (x,r) = \mathcal R _{free}(x)   - h\gamma r - h(1-\gamma)r_k.\]

\begin{equation}
  \mathcal R (x^{\alpha}_{k+1},r^{\alpha}_{k+1}) = \fbox{$\mathcal R^{\alpha}_{k+1} \stackrel{\Delta}{=}  \mathcal R
_{free}(x^{\alpha}_{k+1})  - h\gamma r^{\alpha}_{k+1} - h(1-\gamma)r_k$}\label{eq:rfree-1}
\end{equation}

\[  \mathcal R
_{free}(x^{\alpha}_{k+1},r^{\alpha}_{k+1} )=\fbox{$ \mathcal R _{free k+1} ^{\alpha} \stackrel{\Delta}{=}  M(x^{\alpha}_{k+1} - x_{k}) -h\theta f( x^{\alpha}_{k+1} , t_{k+1}) - h(1-\theta)f(x_k,t_k)$}\]
 
The computation of the Jacobian of $\mathcal R$ with respect to $x$, denoted by $   W^{\alpha}_{k+1}$ leads to 
\begin{equation}
   \label{eq:NL9}
   \begin{array}{l}
    W^{\alpha}_{k+1} \stackrel{\Delta}{=} \nabla_{x} \mathcal R (x^{\alpha}_{k+1},r^{\alpha}_{k+1})= M - h  \theta \nabla_{x} f(  x^{\alpha}_{k+1}, t_{k+1} ).\\
 \end{array}
\end{equation}
At each time--step, we have to solve the following linearized problem,
\begin{equation}
   \label{eq:NL10}
    \mathcal R^{\alpha}_{k+1} + W^{\alpha}_{k+1} (x^{\alpha+1}_{k+1} -
    x^{\alpha}_{k+1}) - h \gamma (r^{\alpha+1}_{k+1} - r^{\alpha}_{k+1} )  =0 ,
\end{equation}
By using (\ref{eq:rfree-1}), we get
\begin{equation}
  \label{eq:rfree-2}
  \mathcal R
_{free}(x^{\alpha}_{k+1},r^{\alpha}_{k+1} )  - h\gamma r^{\alpha+1}_{k+1} - h(1-\gamma)r_k  + W^{\alpha}_{k+1} (x^{\alpha+1}_{k+1} -
    x^{\alpha}_{k+1})  =0 
\end{equation}

%\fbox
{
  \begin{equation}
    \boxed{ x^{\alpha+1}_{k+1} = h(W^{\alpha}_{k+1})^{-1}r^{\alpha+1}_{k+1} +x^\alpha_{free}}
  \end{equation}
}
with :
\begin{equation}
  \boxed{x^\alpha_{free}\stackrel{\Delta}{=}x^{\alpha}_{k+1}-(W^{\alpha}_{k+1})^{-1}\mathcal (R_{freek+1}^{\alpha} \textcolor{red}{- h(1-\gamma) r_k})\label{eq:rfree-12}}
\end{equation}

The matrix $W$ is clearly non singular for small $h$.




% that is

% \begin{equation}
%    \begin{array}{l}
%  h \gamma  r^{\alpha+1}_{k+1} = r_c + W^{\alpha}_{k+1} x^{\alpha+1}_{k+1}
%  .\label{eq:NL11} 
%  \end{array}
% \end{equation}
% with 
% \begin{equation}
%    \begin{array}{l}
% r_c \stackrel{\Delta}{=} h \gamma r^{\alpha}_{k+1} - W^{\alpha}_{k+1} x^{\alpha}_{k+1} + \mathcal R
% ^{\alpha}_{k+1}=- W^{\alpha}_{k+1} x^{\alpha}_{k+1} + \mathcal R_{free k+1} ^{\alpha} - h(1-\gamma)r_k\\ \\
% \end{array}
% \end{equation}
% \begin{equation}
%    \begin{array}{l}
% \mathcal R ^{\alpha}_{k+1}=M( x^{\alpha}_{k+1} - x_k) -h \theta f(x^{\alpha}_{k+1})-h(1-\theta)f(x_k)
% - h \gamma r^{\alpha}_{k+1} -h(1- \gamma)r_k
%  \end{array}
%    \end{equation}
% \[x^{\alpha+1}_{k+1} = h(W^{\alpha}_{k+1})^{-1}r^{\alpha+1}_{k+1} +(W^{\alpha}_{k+1})^{-1}(\mathcal
% R_{free k+1} ^{\alpha})+x^{\alpha}_{k+1}\]


Due to the fact that  two of the  studied classes of systems that are studied in this paper are affine functions in terms of $f$ and $g$, we propose to solve the "one--step nonsmooth problem'' (\ref{eq:toto1}) by performing an external Newton linearization.

 \paragraph{Newton's linearization of the first line of~(\ref{eq:toto1})} The first line of the  problem~(\ref{eq:toto1}) can be written under the form of a residue $\mathcal R$ depending only on $x_{k+1}$ and $r_{k+1}$ such that 
\begin{equation}
  \label{eq:NL3}
  \mathcal R (x_{k+1},r _{k+1}) =0
\end{equation}
with 
\begin{equation}
\mathcal R(x,r) = M(x - x_{k}) -h\theta f( x , t_{k+1}) - h(1-\theta)f(x_k,t_k) - h\gamma r
- h(1-\gamma)r_k.
\end{equation}
The solution of this system of nonlinear equations is sought as a limit of the sequence $\{ x^{\alpha}_{k+1},r^{\alpha}_{k+1} \}_{\alpha \in \NN}$ such that
 \begin{equation}
   \label{eq:NL7}
   \begin{cases}
     x^{0}_{k+1} = x_k \\ \\
     r^{0}_{k+1} = r_k \\ \\
     \mathcal R_L( x^{\alpha+1}_{k+1},r^{\alpha+1}_{k+1}) = \mathcal
     R(x^{\alpha}_{k+1},r^{\alpha}_{k+1})  + \left[ \nabla_{x} \mathcal
     R(x^{\alpha}_{k+1},r^{\alpha}_{k+1})\right] (x^{\alpha+1}_{k+1}-x^{\alpha}_{k+1} ) +
     \left[ \nabla_{r} \mathcal R(x^{\alpha}_{k+1},r^{\alpha}_{k+1})\right] (r^{\alpha+1}_{k+1} - r^{\alpha}_{k+1} ) =0
 \end{cases}
\end{equation}
\begin{ndrva}
  What about $r^0_{k+1}$ ?
\end{ndrva}

The residu \free $\mathcal R _{\free}$ is also defined (useful for implementation only):
\[\mathcal R _{\free}(x) \stackrel{\Delta}{=}  M(x - x_{k}) -h\theta f( x , t_{k+1}) - h(1-\theta)f(x_k,t_k),\]
which yields
\[\mathcal R (x,r) = \mathcal R _{\free}(x)   - h\gamma r - h(1-\gamma)r_k.\]

\begin{equation}
  \mathcal R (x^{\alpha}_{k+1},r^{\alpha}_{k+1}) = \fbox{$\mathcal R^{\alpha}_{k+1} \stackrel{\Delta}{=}  \mathcal R
_{\free}(x^{\alpha}_{k+1})  - h\gamma r^{\alpha}_{k+1} - h(1-\gamma)r_k$}\label{eq:rfree-1}
\end{equation}

\[  \mathcal R
_{\free}(x^{\alpha}_{k+1},r^{\alpha}_{k+1} )=\fbox{$ \mathcal R _{\free, k+1} ^{\alpha} \stackrel{\Delta}{=}  M(x^{\alpha}_{k+1} - x_{k}) -h\theta f( x^{\alpha}_{k+1} , t_{k+1}) - h(1-\theta)f(x_k,t_k)$}\]
 
The computation of the Jacobian of $\mathcal R$ with respect to $x$, denoted by $   W^{\alpha}_{k+1}$ leads to 
\begin{equation}
   \label{eq:NL9}
   \begin{array}{l}
    W^{\alpha}_{k+1} \stackrel{\Delta}{=} \nabla_{x} \mathcal R (x^{\alpha}_{k+1},r^{\alpha}_{k+1})= M - h  \theta \nabla_{x} f(  x^{\alpha}_{k+1}, t_{k+1} ).\\
 \end{array}
\end{equation}
At each time--step, we have to solve the following linearized problem,
\begin{equation}
   \label{eq:NL10}
    \mathcal R^{\alpha}_{k+1} + W^{\alpha}_{k+1} (x^{\alpha+1}_{k+1} -
    x^{\alpha}_{k+1}) - h \gamma (r^{\alpha+1}_{k+1} - r^{\alpha}_{k+1} )  =0 ,
\end{equation}
By using (\ref{eq:rfree-1}), we get
\begin{equation}
  \label{eq:rfree-2}
  \mathcal R
_{\free}(x^{\alpha}_{k+1},r^{\alpha}_{k+1} )  - h\gamma r^{\alpha+1}_{k+1} - h(1-\gamma)r_k  + W^{\alpha}_{k+1} (x^{\alpha+1}_{k+1} -
    x^{\alpha}_{k+1})  =0 
\end{equation}

%\fbox
{
  \begin{equation}
    \label{eq:rfree-11}
    \boxed{ x^{\alpha+1}_{k+1} = h\gamma (W^{\alpha}_{k+1})^{-1}r^{\alpha+1}_{k+1} +x^\alpha_{\free}}
  \end{equation}
}
with :
\begin{equation}
  \label{eq:rfree-12}
  \boxed{x^\alpha_{\free}\stackrel{\Delta}{=}x^{\alpha}_{k+1}-(W^{\alpha}_{k+1})^{-1}\mathcal (R_{\free,k+1}^{\alpha} \textcolor{red}{- h(1-\gamma) r_k})}
\end{equation}

The matrix $W$ is clearly non singular for small $h$.




% that is

% \begin{equation}
%    \begin{array}{l}
%  h \gamma  r^{\alpha+1}_{k+1} = r_c + W^{\alpha}_{k+1} x^{\alpha+1}_{k+1}
%  .\label{eq:NL11} 
%  \end{array}
% \end{equation}
% with 
% \begin{equation}
%    \begin{array}{l}
% r_c \stackrel{\Delta}{=} h \gamma r^{\alpha}_{k+1} - W^{\alpha}_{k+1} x^{\alpha}_{k+1} + \mathcal R
% ^{\alpha}_{k+1}=- W^{\alpha}_{k+1} x^{\alpha}_{k+1} + \mathcal R_{\free k+1} ^{\alpha} - h(1-\gamma)r_k\\ \\
% \end{array}
% \end{equation}
% \begin{equation}
%    \begin{array}{l}
% \mathcal R ^{\alpha}_{k+1}=M( x^{\alpha}_{k+1} - x_k) -h \theta f(x^{\alpha}_{k+1})-h(1-\theta)f(x_k)
% - h \gamma r^{\alpha}_{k+1} -h(1- \gamma)r_k
%  \end{array}
%    \end{equation}
% \[x^{\alpha+1}_{k+1} = h(W^{\alpha}_{k+1})^{-1}r^{\alpha+1}_{k+1} +(W^{\alpha}_{k+1})^{-1}(\mathcal
% R_{\free k+1} ^{\alpha})+x^{\alpha}_{k+1}\]

 \paragraph{Newton's linearization of the second  line of~(\ref{eq:toto1})}
The same operation is performed with the second equation of (\ref{eq:toto1})
\begin{equation}
  \begin{array}{l}
    \mathcal R_y(x,y,\lambda)=y-h(t_{k+1},x,\lambda) =0\\ \\
  \end{array}
\end{equation}
which is linearized as
\begin{equation}
  \label{eq:NL9}
  \begin{array}{l}
    \mathcal R_{Ly}(x^{\alpha+1}_{k+1},y^{\alpha+1}_{k+1},\lambda^{\alpha+1}_{k+1}) = \mathcal
    R_{y}(x^{\alpha}_{k+1},y^{\alpha}_{k+1},\lambda^{\alpha}_{k+1}) +
    (y^{\alpha+1}_{k+1}-y^{\alpha}_{k+1})- \\[2mm] \qquad  \qquad \qquad \qquad  \qquad \qquad
    C^{\alpha}_{k+1}(x^{\alpha+1}_{k+1}-x^{\alpha}_{k+1}) - D^{\alpha}_{k+1}(\lambda^{\alpha+1}_{k+1}-\lambda^{\alpha}_{k+1})=0
  \end{array}
\end{equation}

This leads to the following linear equation
\begin{equation}
  \boxed{y^{\alpha+1}_{k+1} =  y^{\alpha}_{k+1}
  -\mathcal R^{\alpha}_{yk+1}+ \\
  C^{\alpha}_{k+1}(x^{\alpha+1}_{k+1}-x^{\alpha}_{k+1}) +
  D^{\alpha}_{k+1}(\lambda^{\alpha+1}_{k+1}-\lambda^{\alpha}_{k+1})}. \label{eq:NL11y}
\end{equation}
with,
\begin{equation}
     \begin{array}{l}
  C^{\alpha}_{k+1} = \nabla_xh(t_{k+1}, x^{\alpha}_{k+1},\lambda^{\alpha}_{k+1} ) \\ \\
  D^{\alpha}_{k+1} = \nabla_{\lambda}h(t_{k+1}, x^{\alpha}_{k+1},\lambda^{\alpha}_{k+1})
 \end{array}
\end{equation}
and
\begin{equation}\fbox{$
\mathcal R^{\alpha}_{yk+1} \stackrel{\Delta}{=} y^{\alpha}_{k+1} - h(x^{\alpha}_{k+1},\lambda^{\alpha}_{k+1})$}
 \end{equation}
 \paragraph{Newton's linearization of the third  line of~(\ref{eq:toto1})}
The same operation is performed with the third equation of (\ref{eq:toto1})
\begin{equation}
  \begin{array}{l}
    \mathcal R_r(r,x,\lambda)=r-g(t_{k+1},x,\lambda) =0\\ \\  \end{array}
\end{equation}
which is linearized as
\begin{equation}
  \label{eq:NL9}
  \begin{array}{l}
      \mathcal R_{Lr}(r^{\alpha+1}_{k+1},x^{\alpha+1}_{k+1},\lambda^{\alpha+1}_{k+1}) = \mathcal
      R_{rk+1}^{\alpha} + (r^{\alpha+1}_{k+1} - r^{\alpha}_{k+1}) -
      K^{\alpha}_{k+1}(x^{\alpha+1}_{k+1} - x^{\alpha}_{k+1})- B^{\alpha}_{k+1}(\lambda^{\alpha+1}_{k+1} -
      \lambda^{\alpha}_{k+1})=0
    \end{array}
  \end{equation}
\begin{equation}
  \label{eq:rrL}
  \begin{array}{l}
    \boxed{r^{\alpha+1}_{k+1} = g(t_{k+1},x ^{\alpha}_{k+1},\lambda ^{\alpha}_{k+1}) +
      K^{\alpha}_{k+1}(x^{\alpha+1}_{k+1} - x^{\alpha}_{k+1})
      + B^{\alpha}_{k+1}(\lambda^{\alpha+1}_{k+1} - \lambda^{\alpha}_{k+1})
    }       
  \end{array}
\end{equation}
with,
\begin{equation}
     \begin{array}{l}
  K^{\alpha}_{k+1} = \nabla_xg(t_{k+1},x^{\alpha}_{k+1},\lambda ^{\alpha}_{k+1})  \\ \\
  B^{\alpha}_{k+1} = \nabla_{\lambda}g(t_{k+1},x^{\alpha}_{k+1},\lambda ^{\alpha}_{k+1})
 \end{array}
\end{equation}
and the  residue for $r$:
\begin{equation}
\boxed{\mathcal
      R_{rk+1}^{\alpha} = r^{\alpha}_{k+1} - g(t_{k+1},x^{\alpha}_{k+1},\lambda ^{\alpha}_{k+1})}
  \end{equation}


\paragraph{Reduction to a linear relation between  $x^{\alpha+1}_{k+1}$ and
$\lambda^{\alpha+1}_{k+1}$}

Inserting (\ref{eq:rrL}) into~(\ref{eq:rfree-11}), we get the following linear relation between $x^{\alpha+1}_{k+1}$ and
$\lambda^{\alpha+1}_{k+1}$, 

\begin{equation}
   \begin{array}{l}
     x^{\alpha+1}_{k+1} = h\gamma(W^{\alpha}_{k+1} )^{-1}\left[g(t_{k+1},x^{\alpha}_{k+1},\lambda^{\alpha}_{k+1}) +
    B^{\alpha}_{k+1} (\lambda^{\alpha+1}_{k+1} - \lambda^{\alpha}_{k+1})+K^{\alpha}_{k+1}
    (x^{\alpha+1}_{k+1} - x^{\alpha}_{k+1}) \right ] +x^\alpha_{\free}
\end{array}
\end{equation}
that is 
\begin{equation}
  \begin{array}{l}
    (I-h \gamma (W^{\alpha}_{k+1})^{-1}K^{\alpha}_{k+1})x^{\alpha+1}_{k+1}=x_p + h \gamma (W^{\alpha}_{k+1})^{-1}    B^{\alpha}_{k+1} \lambda^{\alpha+1}_{k+1}
   \end{array}
\end{equation}
with 
\begin{equation}
  \boxed{x_p \stackrel{\Delta}{=}  h\gamma(W^{\alpha}_{k+1} )^{-1}\left[g(t_{k+1},x^{\alpha}_{k+1},\lambda^{\alpha}_{k+1}) 
    -B^{\alpha}_{k+1} (\lambda^{\alpha}_{k+1})-K^{\alpha}_{k+1} (x^{\alpha}_{k+1}) \right ] +x^\alpha_{\free}}
\end{equation}



Let us  define the new matrix
\begin{equation}
\tilde K^{\alpha}_{k+1}=(I-h \gamma (W^{\alpha}_{k+1})^{-1} K^{\alpha}_{k+1}).
\label{eq:hatW}
\end{equation}
We get the linear relation
\begin{equation}
  \label{eq:rfree-13}
  \begin{array}{l}
 \boxed{   x^{\alpha+1}_{k+1}\stackrel{\Delta}{=} \tilde K^{\alpha,-1}_{k+1} x_p + \tilde K^{\alpha,-1}_{k+1} \left[ h \gamma (W^{\alpha}_{k+1})^{-1}    B^{\alpha}_{k+1} \lambda^{\alpha+1}_{k+1}\right]}
   \end{array}
\end{equation}



\paragraph{Reduction to a linear relation between  $y^{\alpha+1}_{k+1}$ and
$\lambda^{\alpha+1}_{k+1}$}

Inserting (\ref{eq:rfree-13}) into (\ref{eq:NL11y}), we get the following linear relation between $y^{\alpha+1}_{k+1}$ and $\lambda^{\alpha+1}_{k+1}$, 
\begin{equation}
   \begin{array}{l}
 y^{\alpha+1}_{k+1} = y_p + \left[ h \gamma C^{\alpha}_{k+1} (\tilde K^{\alpha}_{k+1})^{-1}( W^{\alpha}_{k+1})^{-1}  B^{\alpha}_{k+1} + D^{\alpha}_{k+1} \right]\lambda^{\alpha+1}_{k+1}
   \end{array}
\end{equation}
with 
\begin{equation}\boxed{
y_p = y^{\alpha}_{k+1} -\mathcal R^{\alpha}_{yk+1} + C^{\alpha}_{k+1}(x_q) -
D^{\alpha}_{k+1} \lambda^{\alpha}_{k+1} }
\end{equation}
\textcolor{red}{
  \begin{equation}
   \boxed{ x_q=(\tilde K^{\alpha}_{k+1})^{-1}x_p -x^{\alpha}_{k+1}\label{eq:xqq}}
  \end{equation}
}







% \paragraph{With $\gamma =1$:}
% \[(W^{\alpha}_{k+1} )x^{\alpha+1}_{k+1}= hr^{\alpha+1}_{k+1}- \mathcal R_{\free, k+1} ^{\alpha}+W^{\alpha}_{k+1}x^{\alpha}_{k+1}\]
% \[x^{\alpha+1}_{k+1}= h( W^{\alpha}_{k+1})^{-1}r^{\alpha+1}_{k+1}-
% ( W^{\alpha}_{k+1})^{-1} \mathcal R_{\free k+1} ^{\alpha}+x^{\alpha}_{k+1}\]
% \[x^{\alpha+1}_{k+1}= h( W^{\alpha}_{k+1})^{-1}r^{\alpha+1}_{k+1}+x_{\free}\]
% with, using \ref{}
% \begin{equation}
% x_p-x^{\alpha}_{k+1}=h(
% W^{\alpha}_{k+1})^{-1}(g(x^{\alpha}_{k+1},\lambda^{\alpha}_{k+1},t_{k+1})-B^{\alpha}_{k+1}
% \lambda^{\alpha}_{k+1}-K^{\alpha}_{k+1} x^{\alpha}_{k}))+\tilde x_{\free}
% \end{equation}
% \[    \tilde x_{\free}= -( W^{\alpha}_{k+1})^{-1} \mathcal R _{\free k+1} ^{\alpha} \]
%       \[x_{\free} = \tilde x_{\free} + x^{\alpha}_{k+1}=\fbox{$- W^{-1}R_{\free k+1} ^{\alpha} + x^{\alpha}_{k+1}$}\]
% \[ \fbox{$x_p  = x_{\free} + h ( W^{\alpha}_{k+1})^{-1}( g(x ^{\alpha}_{k+1},\lambda ^{\alpha}_{k+1},t_{k+1}) -
%       B^{\alpha}_{k+1} \lambda^{\alpha}_{k+1}-K^{\alpha}_{k+1} x^{\alpha}_{k+1} )$} \]




\paragraph{Mixed linear complementarity problem (MLCP)}To summarize, the problem to be solved in each Newton iteration is:\\{
  \begin{minipage}[l]{1.0\linewidth}
    \begin{equation}
      \begin{cases}
      \begin{array}[l]{l}
        y^{\alpha+1}_{k+1} =   W_{mlcpk+1}^{\alpha}  \lambda^{\alpha+1}_{k+1} + b^{\alpha}_{k+1}
        \\ \\
        -y^{\alpha+1}_{k+1} \in N_{[l,u]}(\lambda^{\alpha+1}_{k+1} ). 
      \end{array}
      \label{eq:NL14}
      \end{cases}
    \end{equation}
  \end{minipage}
}
with $W_{mlcpk+1}\in \RR^{m\times m}$ and $b\in\RR^{m}$ defined by
\begin{equation}
  \label{eq:NL15}
 \begin{array}[l]{l}
   W_{mlcpk+1}^{\alpha} = h \gamma C^{\alpha}_{k+1} (\tilde K^{\alpha}_{k+1})^{-1} (W^{\alpha}_{k+1})^{-1}  B^{\alpha}_{k+1} + D^{\alpha}_{k+1} \\
   b^{\alpha}_{k+1} = y_p
\end{array}
\end{equation}

The problem~(\ref{eq:NL14}) is equivalent to a Mixed Linear Complementarity Problem (MLCP) which can be solved under suitable assumptions by many linear complementarity solvers such as pivoting techniques, interior point techniques and splitting/projection strategies. The  reformulation into a standard MLCP follows the same line as for the MCP in the previous section. One obtains,
    \begin{equation}
      \begin{array}[l]{l}
        y^{\alpha+1}_{k+1} =   - W^{\alpha}_{k+1}  \lambda^{\alpha+1}_{k+1} + b^{\alpha}_{k+1}
        \\ \\
        (y^{\alpha+1}_{k+1})_i  = 0 \qquad \textrm{ for } i \in \{ 1..n\}\\[2mm]
        0 \leq  (\lambda^{\alpha+1}_{k+1})_i\perp (y^{\alpha+1}_{k+1})_i \geq 0 \qquad \textrm{ for } i \in \{ n..n+m\}\\
      \end{array}
      \label{eq:MLCP1} 
    \end{equation}




%%% Local Variables: 
%%% mode: latex
%%% TeX-master: "DevNotes"
%%% End:


\section{Newton's linearization of~(\ref{first-DS2})} 




Let us now proceed with the time discretization of (\ref{first-DS2}) by a fully implicit scheme : 
\begin{equation}
  \begin{array}{l}
    \label{eq:mlcp2-toto1-DS2}
     M x_{k+1} = M x_{k} +h\theta f(x_{k+1},t_{k+1})+h(1-\theta) f(x_k,t_k) + h \gamma r(t_{k+1})
     + h(1-\gamma)r(t_k)  \\[2mm]
     y_{k+1} =  h(t_{k+1},x_{k+1},\lambda _{k+1}) \\[2mm]
     r_{k+1} = g(\lambda_{k+1},t_{k+1})\\[2mm]
  \end{array}
\end{equation}


 \paragraph{Newton's linearization of the first line of~(\ref{eq:mlcp2-toto-DS2})} The first line of the  problem~(\ref{eq:mlcp2-toto-DS2}) can be written under the form of a residue $\mathcal R$ depending only on $x_{k+1}$ and $r_{k+1}$ such that 
\begin{equation}
  \label{eq:mlcp2-NL3}
  \mathcal R (x_{k+1},r _{k+1}) =0
\end{equation}
with $\mathcal R(x,r) = M(x - x_{k}) -h\theta f( x , t_{k+1}) - h(1-\theta)f(x_k,t_k) - h\gamma r
- h(1-\gamma)r_k$.
The solution of this system of nonlinear equations is sought as a limit of the sequence $\{ x^{\alpha}_{k+1},r^{\alpha}_{k+1} \}_{\alpha \in \NN}$ such that
 \begin{equation}
   \label{eq:mlcp2-NL7}
   \begin{cases}
     x^{0}_{k+1} = x_k \\ \\
     \mathcal R_L( x^{\alpha+1}_{k+1},r^{\alpha+1}_{k+1}) = \mathcal
     R(x^{\alpha}_{k+1},r^{\alpha}_{k+1})  + \left[ \nabla_{x} \mathcal
     R(x^{\alpha}_{k+1},r^{\alpha}_{k+1})\right] (x^{\alpha+1}_{k+1}-x^{\alpha}_{k+1} ) +
     \left[ \nabla_{r} \mathcal R(x^{\alpha}_{k+1},r^{\alpha}_{k+1})\right] (r^{\alpha+1}_{k+1} - r^{\alpha}_{k+1} ) =0
 \end{cases}
\end{equation}
\begin{ndrva}
  What about $r^0_{k+1}$ ?
\end{ndrva}

The residu free is also defined (useful for implementation only):
\[\mathcal R _{free}(x) \stackrel{\Delta}{=}  M(x - x_{k}) -h\theta f( x , t_{k+1}) - h(1-\theta)f(x_k,t_k),\]
which yields
\[\mathcal R (x,r) = \mathcal R _{free}(x)   - h\gamma r - h(1-\gamma)r_k.\]

\begin{equation}
  \mathcal R (x^{\alpha}_{k+1},r^{\alpha}_{k+1}) = \fbox{$\mathcal R^{\alpha}_{k+1} \stackrel{\Delta}{=}  \mathcal R
_{free}(x^{\alpha}_{k+1},r^{\alpha}_{k+1} )  - h\gamma r^{\alpha}_{k+1} - h(1-\gamma)r_k$}\label{eq:mlcp2-rfree-1}
\end{equation}

\[  \mathcal R
_{free}(x^{\alpha}_{k+1},r^{\alpha}_{k+1} )=\fbox{$ \mathcal R _{free k+1} ^{\alpha} \stackrel{\Delta}{=}  M(x^{\alpha}_{k+1} - x_{k}) -h\theta f( x^{\alpha}_{k+1} , t_{k+1}) - h(1-\theta)f(x_k,t_k)$}\]
 
The computation of the Jacobian of $\mathcal R$ with respect to $x$, denoted by $   W^{\alpha}_{k+1}$ leads to 
\begin{equation}
   \label{eq:mlcp2-NL9}
   \begin{array}{l}
    W^{\alpha}_{k+1} \stackrel{\Delta}{=} \nabla_{x} \mathcal R (x^{\alpha}_{k+1},r^{\alpha}_{k+1})= M - h  \theta \nabla_{x} f(  x^{\alpha}_{k+1}, t_{k+1} ).\\
 \end{array}
\end{equation}
At each time--step, we have to solve the following linearized problem,
\begin{equation}
   \label{eq:mlcp2-NL10}
    \mathcal R^{\alpha}_{k+1} + W^{\alpha}_{k+1} (x^{\alpha+1}_{k+1} -
    x^{\alpha}_{k+1}) - h \gamma (r^{\alpha+1}_{k+1} - r^{\alpha}_{k+1} )  =0 ,
\end{equation}
By using (\ref{eq:mlcp2-rfree-1}), we get
\begin{equation}
  \label{eq:mlcp2-rfree-2}
  \mathcal R
_{free}(x^{\alpha}_{k+1},r^{\alpha}_{k+1} )  - h\gamma r^{\alpha+1}_{k+1} - h(1-\gamma)r_k  + W^{\alpha}_{k+1} (x^{\alpha+1}_{k+1} -
    x^{\alpha}_{k+1})  =0 
\end{equation}

%\fbox
{
  \begin{equation}
    \boxed{ x^{\alpha+1}_{k+1} = h(W^{\alpha}_{k+1})^{-1}r^{\alpha+1}_{k+1} +x^\alpha_{free}}
  \end{equation}
}
with :
\begin{equation}
  \boxed{x^\alpha_{free}\stackrel{\Delta}{=}x^{\alpha}_{k+1}-(W^{\alpha}_{k+1})^{-1}\mathcal (R_{freek+1}^{\alpha} \textcolor{red}{- h(1-\gamma) r_k})\label{eq:mlcp2-rfree-12}}
\end{equation}

The matrix $W$ is clearly non singular for small $h$.




% that is

% \begin{equation}
%    \begin{array}{l}
%  h \gamma  r^{\alpha+1}_{k+1} = r_c + W^{\alpha}_{k+1} x^{\alpha+1}_{k+1}
%  .\label{eq:mlcp2-NL11} 
%  \end{array}
% \end{equation}
% with 
% \begin{equation}
%    \begin{array}{l}
% r_c \stackrel{\Delta}{=} h \gamma r^{\alpha}_{k+1} - W^{\alpha}_{k+1} x^{\alpha}_{k+1} + \mathcal R
% ^{\alpha}_{k+1}=- W^{\alpha}_{k+1} x^{\alpha}_{k+1} + \mathcal R_{free k+1} ^{\alpha} - h(1-\gamma)r_k\\ \\
% \end{array}
% \end{equation}
% \begin{equation}
%    \begin{array}{l}
% \mathcal R ^{\alpha}_{k+1}=M( x^{\alpha}_{k+1} - x_k) -h \theta f(x^{\alpha}_{k+1})-h(1-\theta)f(x_k)
% - h \gamma r^{\alpha}_{k+1} -h(1- \gamma)r_k
%  \end{array}
%    \end{equation}
% \[x^{\alpha+1}_{k+1} = h(W^{\alpha}_{k+1})^{-1}r^{\alpha+1}_{k+1} +(W^{\alpha}_{k+1})^{-1}(\mathcal
% R_{free k+1} ^{\alpha})+x^{\alpha}_{k+1}\]


 \paragraph{Newton's linearization of the second  line of~(\ref{eq:mlcp2-toto1-DS2})}
The same operation is performed with the second equation of (\ref{eq:mlcp2-toto1-DS2})
\begin{equation}
  \begin{array}{l}
    \mathcal R_y(x,y,\lambda)=y-h(t_{k+1},x,\lambda) =0\\ \\
  \end{array}
\end{equation}
which is linearized as
\begin{equation}
  \label{eq:mlcp2-NL9}
  \begin{array}{l}
    \mathcal R_{Ly}(x^{\alpha+1}_{k+1},y^{\alpha+1}_{k+1},\lambda^{\alpha+1}_{k+1}) = \mathcal
    R_{y}(x^{\alpha}_{k+1},y^{\alpha}_{k+1},\lambda^{\alpha}_{k+1}) +
    (y^{\alpha+1}_{k+1}-y^{\alpha}_{k+1})- \\[2mm] \qquad  \qquad \qquad \qquad  \qquad \qquad
    C^{\alpha}_{k+1}(x^{\alpha+1}_{k+1}-x^{\alpha}_{k+1}) - D^{\alpha}_{k+1}(\lambda^{\alpha+1}_{k+1}-\lambda^{\alpha}_{k+1})=0
  \end{array}
\end{equation}

This leads to the following linear equation
\begin{equation}
  \boxed{y^{\alpha+1}_{k+1} =  y^{\alpha}_{k+1}
  -\mathcal R^{\alpha}_{yk+1}+ \\
  C^{\alpha}_{k+1}(x^{\alpha+1}_{k+1}-x^{\alpha}_{k+1}) +
  D^{\alpha}_{k+1}(\lambda^{\alpha+1}_{k+1}-\lambda^{\alpha}_{k+1})}. \label{eq:mlcp2-NL11y}
\end{equation}
with,
\begin{equation}
     \begin{array}{l}
  C^{\alpha}_{k+1} = \nabla_xh(t_{k+1}, x^{\alpha}_{k+1},\lambda^{\alpha}_{k+1} ) \\ \\
  D^{\alpha}_{k+1} = \nabla_{\lambda}h(t_{k+1}, x^{\alpha}_{k+1},\lambda^{\alpha}_{k+1})
 \end{array}
\end{equation}
and
\begin{equation}\fbox{$
\mathcal R^{\alpha}_{yk+1} \stackrel{\Delta}{=} y^{\alpha}_{k+1} - h(x^{\alpha}_{k+1},\lambda^{\alpha}_{k+1})$}
 \end{equation}
 \paragraph{Newton's linearization of the third  line of~(\ref{eq:mlcp2-toto1-DS2})}
The same operation is performed with the third equation of (\ref{eq:mlcp2-toto1-DS2})
\begin{equation}
  \begin{array}{l}
    \mathcal R_r(r,x,\lambda)=r-g(\lambda,t_{k+1}) =0\\ \\  \end{array}
\end{equation}
which is linearized as
\begin{equation}
  \label{eq:mlcp2-NL9}
  \begin{array}{l}
      \mathcal R_{L\lambda}(r^{\alpha+1}_{k+1},x^{\alpha+1}_{k+1},\lambda^{\alpha+1}_{k+1}) = \mathcal
      R_{rk+1}^{\alpha} + (r^{\alpha+1}_{k+1} - r^{\alpha}_{k+1}) - B^{\alpha}_{k+1}(\lambda^{\alpha+1}_{k+1} -
      \lambda^{\alpha}_{k+1})=0
    \end{array}
  \end{equation}
\begin{equation}
  \label{eq:mlcp2-rrL}
  \begin{array}{l}
    \boxed{r^{\alpha+1}_{k+1} = g(x ^{\alpha}_{k+1},\lambda ^{\alpha}_{k+1},t_{k+1}) -B^{\alpha}_{k+1}
      \lambda^{\alpha}_{k+1} + B^{\alpha}_{k+1} \lambda^{\alpha+1}}       
  \end{array}
\end{equation}
with,
\begin{equation}
     \begin{array}{l}
  B^{\alpha}_{k+1} = \nabla_{\lambda}g(x^{\alpha}_{k+1},\lambda ^{\alpha}_{k+1},t_{k+1})
 \end{array}
\end{equation}
and the  residue for $r$:
\begin{equation}
\boxed{\mathcal
      R_{rk+1}^{\alpha} = r^{\alpha}_{k+1} - g(\lambda ^{\alpha}_{k+1},t_{k+1})}
  \end{equation}


\paragraph{Reduction to a linear relation between  $x^{\alpha+1}_{k+1}$ and
$\lambda^{\alpha+1}_{k+1}$}

Inserting (\ref{eq:mlcp2-rrL}) into~(\ref{eq:mlcp2-rfree-12}), we get the following linear relation between $x^{\alpha+1}_{k+1}$ and
$\lambda^{\alpha+1}_{k+1}$, 

\begin{equation}
   \begin{array}{l}
     x^{\alpha+1}_{k+1} = h\gamma(W^{\alpha}_{k+1} )^{-1}\left[g(x^{\alpha}_{k+1},\lambda^{\alpha}_{k+1},t_{k+1}) +
    B^{\alpha}_{k+1} (\lambda^{\alpha+1}_{k+1} - \lambda^{\alpha}_{k+1}) \right ] +x^\alpha_{free}
\end{array}
\end{equation}
that is 
\begin{equation}
  \begin{array}{l}
   x^{\alpha+1}_{k+1} =x_p + h \gamma (W^{\alpha}_{k+1})^{-1}    B^{\alpha}_{k+1} \lambda^{\alpha+1}_{k+1}
   \end{array}
\end{equation}
with 
\begin{equation}
  \boxed{x_p \stackrel{\Delta}{=}  h\gamma(W^{\alpha}_{k+1} )^{-1}\left[g(x^{\alpha}_{k+1},\lambda^{\alpha}_{k+1},t_{k+1}) +
    -B^{\alpha}_{k+1} (\lambda^{\alpha}_{k+1}) \right ] +x^\alpha_{free}}
\end{equation}


We get the linear relation
\begin{equation}
  \label{eq:mlcp2-rfree-13}
  \begin{array}{l}
 \boxed{   x^{\alpha+1}_{k+1}\stackrel{\Delta}{=} x_p + \left[ h \gamma (W^{\alpha}_{k+1})^{-1}    B^{\alpha}_{k+1} \lambda^{\alpha+1}_{k+1}\right]}
   \end{array}
\end{equation}




\paragraph{Reduction to a linear relation between  $y^{\alpha+1}_{k+1}$ and
$\lambda^{\alpha+1}_{k+1}$}

Inserting (\ref{eq:mlcp2-rfree-13}) into (\ref{eq:mlcp2-NL11y}), we get the following linear relation between $y^{\alpha+1}_{k+1}$ and $\lambda^{\alpha+1}_{k+1}$, 
\begin{equation}
   \begin{array}{l}
 y^{\alpha+1}_{k+1} = y_p + \left[ h  C^{\alpha}_{k+1} ( W^{\alpha}_{k+1})^{-1}  B^{\alpha}_{k+1} + D^{\alpha}_{k+1} \right]\lambda^{\alpha+1}_{k+1}
   \end{array}
\end{equation}
with 
\begin{equation}\boxed{
y_p = y^{\alpha}_{k+1} -\mathcal R^{\alpha}_{yk+1} + C^{\alpha}_{k+1}(x_q) -
D^{\alpha}_{k+1} \lambda^{\alpha}_{k+1} }
\end{equation}
\textcolor{red}{
  \begin{equation}
    \boxed{ x_q= x_p - x^{\alpha}_{k+1}\label{eq:mlcp2-xqq}}
  \end{equation}
}





\clearpage


%%% Local Variables: 
%%% mode: latex
%%% TeX-master: "DevNotes"
%%% End: 
 

\subsection{Time--discretization of the linear case~(\ref{eq:quatre}) } 

Let us now proceed with the time discretization of (\ref{first-DS3}) by a fully implicit scheme : 
\begin{equation}
  \begin{array}{l}
    \label{eq:toto1-DS3}
     M x^{\alpha+1}_{k+1} = M x_{k} +h\theta A x^{\alpha+1}_{k+1}+h(1-\theta) A x_k + h \gamma r^{\alpha+1}_{k+1}+ h(1-\gamma)r(t_k)  +hb\\[2mm]
     y^{\alpha+1}_{k+1} =  C x^{\alpha+1}_{k+1} + D \lambda ^{\alpha+1}_{k+1} +Fz +e\\[2mm]
     r^{\alpha+1}_{k+1} = B \lambda ^{\alpha+1}_{k+1} \\[2mm]
  \end{array}
\end{equation}

\[R_{free} = M(x^{\alpha}_{k+1} - x_{k}) -h\theta A x^{\alpha}_{k+1} - h(1-\theta) A x_k -hb_{k+1} \]
\[R_{free} = W(x^{\alpha}_{k+1} - x_{k}) -h A x_{k} -hb_{k+1} \]

\subsection{Resulting Newton step (only one step)}
suppose:$\gamma =1$
\begin{equation}
  \begin{array}{l}
     (M -h\theta A)x^{\alpha+1}_{k+1} = M x_{k} +h(1-\theta) A x_k + hr^{\alpha+1}_{k+1} + hb\\[2mm]
     y^{\alpha+1}_{k+1} =  C x^{\alpha+1}_{k+1} + D \lambda ^{\alpha+1}_{k+1} +Fz + e \\[2mm]
     r^{\alpha+1}_{k+1} = B \lambda ^{\alpha+1}_{k+1}\\[2mm]
  \end{array}
\end{equation}
that lead to with: $ (M -h\theta A) = W$
\begin{equation}
  \begin{array}{l}
     x^{\alpha+1}_{k+1} = W^{-1}(M x_{k} +h(1-\theta) A x_k + r^{\alpha+1}_{k+1} +hb) = xfree + W^{-1}(r^{\alpha+1}_{k+1})\\[2mm]
     y^{\alpha+1}_{k+1} =  ( D+hCW^{-1}B) \lambda ^{\alpha+1}_{k+1} +Fz + CW^{-1}(M
     x_k+h(1-\theta)Ax_k + hb) +e \\[2mm]
  \end{array}
\end{equation}
with $x_{free} = x^{\alpha}_{k+1} + W^{-1}(-R_{free})= x^{\alpha}_{k+1} - W^{-1}(W(x^{\alpha}_{k+1}
- x_k) -hAx_k-hb_{k+1} )= W^{-1}(Mx_k +h(1-\theta)Ax_k +h b_{k+1})$
\begin{equation}
  \begin{array}{l}
     y^{\alpha+1}_{k+1} =  ( D+hCW^{-1}B) \lambda ^{\alpha+1}_{k+1} +Fz + Cx_{free}+e\\[2mm]
     r^{\alpha+1}_{k+1} = B \lambda ^{\alpha+1}_{k+1}\\[2mm]
  \end{array}
\end{equation}

\subsection{coherence with previous formulation}
\[y_p = y^{\alpha}_{k+1} -\mathcal R^{\alpha}_{yk+1} + C^{\alpha}_{k+1}(x_p -x^{\alpha}_{k+1}) -
D^{\alpha}_{k+1} \lambda^{\alpha}_{k+1} \]
\[y_p = Cx_k + D \lambda _k  + C(\tilde x_{free}) -D \lambda_k +Fz + e\]
\[y_p = Cx_k   + C(\tilde x_{free})  +Fz + e\]
\[y_p = Cx_k   + C(\tilde x_{free})  +Fz + e\]
\[y_p = C(x_{free})  +Fz + e\]

%In the case of the system~(\ref{eq:deux}) with a affine function $f$ or $\theta =0$, the the MLCP matrix $W$ can be computed before the beginning of the time loop saving a lot of computing effort.  In the case of the system (\ref{eq:trois}) with $\theta=\gamma=0$, the MLCP matrix $W$ can be computed before the beginning of the Newton loop.
\clearpage


%%% Local Variables: 
%%% mode: latex
%%% TeX-master: "DevNotes"
%%% End: 
\section{Newton's linearization of~(\ref{first-DS2}) with  (\ref{eq:toto1-ter}) } 

  \begin{equation}
    \begin{array}{l}
      \label{eq:full-toto1-ter}
      M x_{k+1} = M x_{k} +h \theta f(x_{k+1},t_{k+1}) +h(1-\theta)f(x_{k},t_{k}) + h r_{k+\gamma} \\[2mm]
      y_{k+\gamma} =  h(t_{k+\gamma},x_{k+\gamma},\lambda _{k+\gamma}) \\[2mm]
      r_{k+\gamma} = g(\lambda_{k+\gamma},t_{k+\gamma})\\[2mm]
    \end{array}
\end{equation}

 \paragraph{Newton's linearization of the first line of~(\ref{eq:full-toto1-ter})} The first line of the  problem~(\ref{eq:full-toto1}) can be written under the form of a residue $\mathcal R$ depending only on $x_{k+1}$ and $r_{k+\gamma}$ such that 
\begin{equation}
  \label{eq:full-NL3}
  \mathcal R (x_{k+1},r _{k+\gamma}) =0
\end{equation}
with $\mathcal R(x,r) = M(x - x_{k}) -h\theta f( x , t_{k+1}) - h(1-\theta)f(x_k,t_k) - h r $
The solution of this system of nonlinear equations is sought as a limit of the sequence $\{ x^{\alpha}_{k+1},r^{\alpha}_{k+\gamma} \}_{\alpha \in \NN}$ such that
 \begin{equation}
   \label{eq:full-NL7}
   \begin{cases}
     x^{0}_{k+1} = x_k \\ \\
     \mathcal R_L( x^{\alpha+1}_{k+1},r^{\alpha+1}_{k+\gamma}) = \mathcal
     R(x^{\alpha}_{k+1},r^{\alpha}_{k+\gamma})  + \left[ \nabla_{x} \mathcal
     R(x^{\alpha}_{k+1},r^{\alpha}_{k+\gamma})\right] (x^{\alpha+1}_{k+1}-x^{\alpha}_{k+1} ) + \\[2mm]
     \qquad\qquad\qquad\qquad\qquad\qquad\left[ \nabla_{r} \mathcal R(x^{\alpha}_{k+1},r^{\alpha}_{k+\gamma})\right] (r^{\alpha+1}_{k+\gamma} - r^{\alpha}_{k+\gamma} ) =0
 \end{cases}
\end{equation}
\begin{ndrva}
  What about $r^0_{k+\gamma}$ ?
\end{ndrva}

The residu free is also defined (useful for implementation only):
\[\mathcal R _{free}(x) \stackrel{\Delta}{=}  M(x - x_{k}) -h\theta f( x , t_{k+1}) - h(1-\theta)f(x_k,t_k),\]
which yields
\[\mathcal R (x,r) = \mathcal R _{free}(x)   - h r .\]

\begin{equation}
  \mathcal R (x^{\alpha}_{k+1},r^{\alpha}_{k+\gamma}) = \fbox{$\mathcal R^{\alpha}_{k+1} \stackrel{\Delta}{=}  \mathcal R_{free}(x^{\alpha}_{k+1} )  - h r^{\alpha}_{k+\gamma}$}\label{eq:full-rfree-1}
\end{equation}

\[  \mathcal R
_{free}(x^{\alpha}_{k+1} )=\fbox{$ \mathcal R _{free k+1} ^{\alpha} \stackrel{\Delta}{=}  M(x^{\alpha}_{k+1} - x_{k}) -h\theta f( x^{\alpha}_{k+1} , t_{k+1}) - h(1-\theta)f(x_k,t_k)$}\]
 
The computation of the Jacobian of $\mathcal R$ with respect to $x$, denoted by $   W^{\alpha}_{k+1}$ leads to 
\begin{equation}
   \label{eq:full-NL9}
   \begin{array}{l}
    W^{\alpha}_{k+1} \stackrel{\Delta}{=} \nabla_{x} \mathcal R (x^{\alpha}_{k+1})= M - h  \theta \nabla_{x} f(  x^{\alpha}_{k+1}, t_{k+1} ).\\
 \end{array}
\end{equation}
At each time--step, we have to solve the following linearized problem,
\begin{equation}
   \label{eq:full-NL10}
    \mathcal R^{\alpha}_{k+1} + W^{\alpha}_{k+1} (x^{\alpha+1}_{k+1} -
    x^{\alpha}_{k+1}) - h  (r^{\alpha+1}_{k+\gamma} - r^{\alpha}_{k+\gamma} )  =0 ,
\end{equation}
By using (\ref{eq:full-rfree-1}), we get
\begin{equation}
  \label{eq:full-rfree-2}
  \mathcal R _{free}(x^{\alpha}_{k+1})  - h  r^{\alpha+1}_{k+\gamma}   + W^{\alpha}_{k+1} (x^{\alpha+1}_{k+1} -
    x^{\alpha}_{k+1})  =0 
\end{equation}

%\fbox
{
  \begin{equation}
    \boxed{ x^{\alpha+1}_{k+1} = h(W^{\alpha}_{k+1})^{-1}r^{\alpha+1}_{k+1} +x^\alpha_{free}}
  \end{equation}
}
with :
\begin{equation}
  \boxed{x^\alpha_{free}\stackrel{\Delta}{=}x^{\alpha}_{k+1}-(W^{\alpha}_{k+1})^{-1}\mathcal R_{freek+1}^{\alpha} \label{eq:full-rfree-12}}
\end{equation}

The matrix $W$ is clearly non singular for small $h$.


 \paragraph{Newton's linearization of the second  line of~(\ref{eq:full-toto1})}
The same operation is performed with the second equation of (\ref{eq:full-toto1})
\begin{equation}
  \begin{array}{l}
    \mathcal R_y(x,y,\lambda)=y-h(t_{k+\gamma},\gamma x + (1-\gamma) x_k ,\lambda) =0\\ \\
  \end{array}
\end{equation}
which is linearized as
\begin{equation}
  \label{eq:full-NL9}
  \begin{array}{l}
    \mathcal R_{Ly}(x^{\alpha+1}_{k+1},y^{\alpha+1}_{k+\gamma},\lambda^{\alpha+1}_{k+\gamma}) = \mathcal
    R_{y}(x^{\alpha}_{k+1},y^{\alpha}_{k+\gamma},\lambda^{\alpha}_{k+\gamma}) +
    (y^{\alpha+1}_{k+\gamma}-y^{\alpha}_{k+\gamma})- \\[2mm] \qquad  \qquad \qquad \qquad  \qquad \qquad
    \gamma C^{\alpha}_{k+1}(x^{\alpha+1}_{k+1}-x^{\alpha}_{k+1}) - D^{\alpha}_{k+\gamma}(\lambda^{\alpha+1}_{k+\gamma}-\lambda^{\alpha}_{k+\gamma})=0
  \end{array}
\end{equation}

This leads to the following linear equation
\begin{equation}
  \boxed{y^{\alpha+1}_{k+\gamma} =  y^{\alpha}_{k+\gamma}
  -\mathcal R^{\alpha}_{y,k+1}+ \\
  \gamma C^{\alpha}_{k+1}(x^{\alpha+1}_{k+1}-x^{\alpha}_{k+1}) +
  D^{\alpha}_{k+\gamma}(\lambda^{\alpha+1}_{k+\gamma}-\lambda^{\alpha}_{k+\gamma})}. \label{eq:full-NL11y}
\end{equation}
with,
\begin{equation}
     \begin{array}{l}
  C^{\alpha}_{k+\gamma} = \nabla_xh(t_{k+1}, x^{\alpha}_{k+\gamma},\lambda^{\alpha}_{k+\gamma} ) \\ \\
  D^{\alpha}_{k+\gamma} = \nabla_{\lambda}h(t_{k+1}, x^{\alpha}_{k+\gamma},\lambda^{\alpha}_{k+\gamma})
 \end{array}
\end{equation}
and
\begin{equation}\fbox{$
\mathcal R^{\alpha}_{yk+1} \stackrel{\Delta}{=} y^{\alpha}_{k+\gamma} - h(x^{\alpha}_{k+\gamma},\lambda^{\alpha}_{k+\gamma})$}
 \end{equation}
 \paragraph{Newton's linearization of the third  line of~(\ref{eq:full-toto1})}
The same operation is performed with the third equation of (\ref{eq:full-toto1})
\begin{equation}
  \begin{array}{l}
    \mathcal R_r(r,\lambda)=r-g(\lambda,t_{k+1}) =0\\ \\  \end{array}
\end{equation}
which is linearized as
\begin{equation}
  \label{eq:full-NL9}
  \begin{array}{l}
      \mathcal R_{L\lambda}(r^{\alpha+1}_{k+\gamma},\lambda^{\alpha+1}_{k+\gamma}) = \mathcal
      R_{r,k+\gamma}^{\alpha} + (r^{\alpha+1}_{k+\gamma} - r^{\alpha}_{k+\gamma}) - B^{\alpha}_{k+\gamma}(\lambda^{\alpha+1}_{k+\gamma} -
      \lambda^{\alpha}_{k+\gamma})=0
    \end{array}
  \end{equation}
\begin{equation}
  \label{eq:full-rrL}
  \begin{array}{l}
    \boxed{r^{\alpha+1}_{k+\gamma} = g(\lambda ^{\alpha}_{k+\gamma},t_{k+\gamma}) -B^{\alpha}_{k+\gamma}
      \lambda^{\alpha}_{k+\gamma} + B^{\alpha}_{k+\gamma} \lambda^{\alpha+1}_{k+\gamma}}       
  \end{array}
\end{equation}
with,
\begin{equation}
     \begin{array}{l}
  B^{\alpha}_{k+\gamma} = \nabla_{\lambda}g(\lambda ^{\alpha}_{k+\gamma},t_{k+\gamma})
 \end{array}
\end{equation}
and the  residue for $r$:
\begin{equation}
\boxed{\mathcal
      R_{rk+\gamma}^{\alpha} = r^{\alpha}_{k+\gamma} - g(\lambda ^{\alpha}_{k+\gamma},t_{k+\gamma})}
  \end{equation}


\paragraph{Reduction to a linear relation between  $x^{\alpha+1}_{k+1}$ and
$\lambda^{\alpha+1}_{k+\gamma}$}

Inserting (\ref{eq:full-rrL}) into~(\ref{eq:full-rfree-12}), we get the following linear relation between $x^{\alpha+1}_{k+1}$ and
$\lambda^{\alpha+1}_{k+1}$, 

\begin{equation}
   \begin{array}{l}
     x^{\alpha+1}_{k+1} = h(W^{\alpha}_{k+1} )^{-1}\left[g(\lambda^{\alpha}_{k+\gamma},t_{k+\gamma}) +
    B^{\alpha}_{k+\gamma} (\lambda^{\alpha+1}_{k+\gamma} - \lambda^{\alpha}_{k+\gamma}) \right ] +x^\alpha_{free}
\end{array}
\end{equation}
that is 
\begin{equation}
  \begin{array}{l}
\boxed{x^{\alpha+1}_{k+1}=x_p + h (W^{\alpha}_{k+1})^{-1}    B^{\alpha}_{k+\gamma} \lambda^{\alpha+1}_{k+\gamma}}
   \end{array}
  \label{eq:full-rfree-13}
\end{equation}
with 
\begin{equation}
  \boxed{x_p \stackrel{\Delta}{=}  h(W^{\alpha}_{k+1} )^{-1}\left[g(\lambda^{\alpha}_{k+\gamma},t_{k+\gamma}) -B^{\alpha}_{k+\gamma} (\lambda^{\alpha}_{k+\gamma}) \right ] +x^\alpha_{free}}
\end{equation}


\paragraph{Reduction to a linear relation between  $y^{\alpha+1}_{k+\gamma}$ and
$\lambda^{\alpha+1}_{k+\gamma}$}

Inserting (\ref{eq:full-rfree-13}) into (\ref{eq:full-NL11y}), we get the following linear relation between $y^{\alpha+1}_{k+1}$ and $\lambda^{\alpha+1}_{k+1}$, 
\begin{equation}
   \begin{array}{l}
 y^{\alpha+1}_{k+1} = y_p + \left[ h \gamma C^{\alpha}_{k+\gamma} ( W^{\alpha}_{k+1})^{-1}  B^{\alpha}_{k+1} + D^{\alpha}_{k+1} \right]\lambda^{\alpha+1}_{k+1}
   \end{array}
\end{equation}
with 
\begin{equation}
y_p = y^{\alpha}_{k+1} -\mathcal R^{\alpha}_{yk+1} + \gamma C^{\alpha}_{k+1}(x_q) - D^{\alpha}_{k+1} \lambda^{\alpha}_{k+1} 
\end{equation}
that is 
\begin{equation}\boxed{
y_p =  h(x^{\alpha}_{k+\gamma},\lambda^{\alpha}_{k+\gamma}) + \gamma C^{\alpha}_{k+1}(x_q) - D^{\alpha}_{k+1} \lambda^{\alpha}_{k+1} }
\end{equation}
\textcolor{red}{
  \begin{equation}
   \boxed{ x_q=(x_p -x^{\alpha}_{k+1})\label{eq:full-xqq}}
  \end{equation}
}


\paragraph{The linear case}
\begin{equation}
  \begin{array}{lcl}
    y_p &=&  h(x^{\alpha}_{k+\gamma},\lambda^{\alpha}_{k+\gamma}) + \gamma C^{\alpha}_{k+1}(x_q) - D^{\alpha}_{k+1} \lambda^{\alpha}_{k+1}\\
        &=&  C^{\alpha}_{k+1} x^{\alpha}_{k+\gamma} + D^{\alpha}_{k+1}\lambda^{\alpha}_{k+\gamma}  + \gamma C^{\alpha}_{k+1}(x_q) - D^{\alpha}_{k+1} \lambda^{\alpha}_{k+1} \\
        &=& C^{\alpha}_{k+1}  (x^{\alpha}_{k+\gamma} + \gamma x_p - \gamma x^{\alpha}_{k+1} ) \\
        &=& C^{\alpha}_{k+1}  ((1-\gamma) x_{k} + \gamma x_{free} ) \text {since } x_p =x_{free} 
\end{array}
\end{equation}




\paragraph{Implementation details}

For the moment (Feb. 2011), we set $x_q=(1-\gamma) x_{k} + \gamma x_{free} $ in the linear case The nonlinear case is not yet implemented since we need to change the management of {\verb H_alpha } in Relation to be able to compute the mid--point values. things that remain to  do
\begin{itemize}
\item implement the function \texttt{BlockVector  computeg(t,lambda)} and \texttt{SimpleVector computeh(t,x,lambda)} which takes into account the values of the argument and return and vector
\item remove temporary computation in Relation of {\verb Xq, \verb g_alpha and \verb H_alpha }. This should be stored somewhere else. (in the  node of the graph)
\end{itemize}








\clearpage


%%% Local Variables: 
%%% mode: latex
%%% TeX-master: "DevNotes"
%%% End: 


%%% Local Variables: 
%%% mode: latex
%%% TeX-master: "DevNotes"
%%% End: 
